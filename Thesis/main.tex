% !TeX encoding = UTF-8
% !TeX program = pdflatex
% !TeX spellcheck = en_US

\documentclass[LaM,binding=0.6cm]{sapthesis}

\usepackage[utf8]{inputenc}
\usepackage{amsmath}

%%%%%%%%%%Added these packages-Commands%%%%%%%%%%%%%%%%%%
%%%%%%%%%%%%%%%%%%%%%%%%%%%%%%%%%%%%%%%%%%%%%%%%%%%%%%%%%
\usepackage{graphicx}
\usepackage{booktabs}
\usepackage[ruled,vlined,linesnumered]{algorithm2e}
\usepackage{enumitem}
\usepackage{libertine}
\usepackage{amssymb,amsmath,amsfonts,amstext,amsthm}
\usepackage{hyperref}
\usepackage[svgnames]{xcolor}
\usepackage[capitalise,nameinlink]{cleveref}
\hypersetup{colorlinks={true},linkcolor={DarkBlue},citecolor=[named]{DarkGreen}}
\usepackage{tikz}
\newcommand{\gb}[1]{{\color{red}[GB: #1]}}
%%%%%%%%%%%%%%%%End of the added packages-Commands%%%%%%%%%%%%%
%%%%%%%%%%%%%%%%%%%%%%%%%%%%%%%%%%%%%%%%%%%%%%%%%%%%%%%%%%%%%%%%

\RequirePackage[style=alphabetic]{biblatex}
\RequirePackage[nottoc]{tocbibind}
\RequirePackage{bibentry}

\bibliography{cit}


\newcommand{\floor}[3][2]{\left \lfloor\frac{#2}{#3}\right \rfloor}
\newcommand{\ceil}[3][2]{\left \lceil\frac{#2}{#3}\right \rceil}
\DeclareMathOperator*{\argmax}{arg\,max}
\DeclareMathOperator*{\argmin}{arg\,min}

\usepackage{microtype}

\usepackage{hyperref}

%\hypersetup{pdftitle={Thesis Title},pdfauthor={Francesco Montano}}

% Remove in a normal thesis
\usepackage{lipsum}
\usepackage{curve2e}
\definecolor{gray}{gray}{0.4}
\newcommand{\bs}{\textbackslash}

\usepackage{tikz-cd}

% Commands for the titlepage
\title{On the polynomial computation of EFX allocations for 3 agents and 3-valued instances}
\author{Francesco Montano}
\IDnumber{1744183}
\course{Engineering in Computer Science}
\courseorganizer{Facolt\`{a} di Ingegneria dell'informazione,
informatica e statistica}
\AcademicYear{2020/2021}
\copyyear{2021}
\advisor{Prof. Aris Anagnostopoulos}
\coadvisor{Dr. Georgios Birmpas}

\authoremail{montano.1744183@studenti.uniroma1.it}

%\examdate{16 April 2013}
%\examiner{Prof. Nome Cognome}
%\examiner{Prof. Nome Cognome}
%\examiner{Dr. Nome Cognome}
\versiondate{\today}



\begin{document}

\frontmatter

\maketitle


\begin{abstract}
Fair division is the problem of dividing a set of resources among a set of agents in a fair manner. In the past years several fairness criteria have been introduced, with "envy freeness" to be among the most important ones.
Informally, an allocation is envy free when no agent envies the share that another agent got.
The main drawback of the envy-freeness criterion, is that it cannot always be guaranteed in the setting of indivisible items. For this reason, some more relaxed criteria have been introduced by the literature, two of which are envy freeness up to one good (EF1), and a stronger version of the same concept, namely, envy freeness up to any good (EFX).  Whereas there exists a polynomial time algorithm to obtain EF1 allocations for any instance, the EFX criterion remains currently more elusive, as there are only partial
results regarding its existence.
In this thesis, we are making steps towards improving the current state of the art for this problem, by presenting the first polynomial time algorithm for the computation of EFX allocations, under instances with three players and three values with a constraint on the values. 
\end{abstract}


%\begin{acknowledgments}

%\end{acknowledgments}

\tableofcontents


%\chapter{Capitolo non numerato}

%\section*{Paragrafo non numerato}
\mainmatter

\chapter{Introduction}
% History
Fair division is the problem of dividing a set of resources between a set of players in order to satisfy some criterion of fairness. The theory of fair division dates back to the second world war when three Polish mathematicians, Hugo Steinhaus, Bronisław Knaster and Stefan Banach started working on the fair cake-cutting problem to find a generalization for any number of players of the divide and choose algorithm; the three developed the last diminisher procedure.
% Real problem applications
This problem arises in several real scenarios as distributing tasks, dividing goods \cite{DBLP:GhodsiHSSY18-real-scenarios}, frequency allocation, airport traffic management, and the fair and efficient exploitation of Earth Observation Satellites \cite{real-scenarios-2}. There are several application in order to deal with such problems and one of them is the \href{http://www.spliddit.org/}{spliddit} site that returns provably fair solutions for rent sharing, task distributions and other problems. For instance this site for the sharing rent problem, rather than divide the rent equally among the players, he asks to each of them how much they value each room and returns the allocation of the rooms and how much each player has to pay. 

Sometimes other than require some fairness criteria, is required to divide the set of resources between the players by considering also two other aspects: Pareto optimality and truthfulness \cite{wiki-fair-division}. Pareto optimality, or Pareto efficient,  is a concept introduced in Economy and game theory that occurs when the allocation of the resources is such that we cannot obtain a Pareto improvement: a player cannot improve his utility without reducing the utility of another player. Truthfulness is also a concept introduced in game theory in cases in which the players have private information, we say that an asymmetric game is truthful if disclose the private information is a weakly-dominant strategy: a strategy that provides for all the players at least the same utility that they will get by playing other strategies.

Fair division problem can be divided in several manners considering the type of items:
\begin{itemize}
    \item divisible and indivisible items,
    \item homogeneous and heterogeneous items,
    \item goods or bads: items with positive or negative value for the players. An example of bads can be the house chores.
\end{itemize}
In this thesis I am going to approach mainly the problem of indivisible items, but I will first introduce briefly the results obtained in the field of divisible items and then I am going to describe the different type of fairness criteria. 
\section{Divisible Items}
When considering the case of divisible items the main problem that we can think of is the fair cake-cutting problem. In this type of problem, we have to divide a heterogeneous resource between different agents. An example of this problem is when we have to divide a cake with several toppings between different people. In this problem, we consider that the resource can be divided into arbitrary small parts without destroying the value. The players to which we have to assign parts of the resource can have different preferences over the resource, so following the example of a cake with different toppings onto, we can say that the players can have different tastes. The cake is not the only example of this type of problem, land estates, advertisement space and broadcast time are few of other examples in which this type of problem raises. 

As said in the beginning, Hugo Steinhaus, Bronisław Knaster and Stefan Banach have developed an algorithm called diminisher procedure that produces proportional division of an heterogeneous resource between $n$ players, so that each agent receives a part of the complete resource that he values at least $\frac{1}{n}$ of the entire value of the resource. This algorithm is an extension of the cut and choose algorithm for divisible items. The cut and choose algorithm produces EF allocation for two players by letting one player divide the resource into two parts and letting the other player choose his part. We can notice that this simple algorithm produce an EF allocation since the first player will divide the resource in such a way that he is indifferent between the two parts, and the second player will take the part with higher value for him. We can also notice that such type of algorithm is not optimal, let's say that the resource is a cake with two toppings: half chocolate and half vanilla, and that the first player prefers the chocolate, while the second one prefers the vanilla topping. In this case the first player will divide the cake so that each part has half vanilla and half chocolate, while the optimal solution was to give the chocolate half to the first player and the other one to the second one \cite{wiki-fair-division}.   

\section{Indivisible Items}
The subjective theory of value is a theory that proposes the idea that the value of each object is not defined by a property or by the production method, but instead has value determined by the importance that people give to it, for this reason in the fairness theory we tend to consider subjective concepts of fairness\cite{wiki-fair-division}. 
Before introducing some of the most used fairness criterion let's assume that we have a set of $n$ agents $N=\{1,2,\dots,n\}$ and a set $M$ of $m$ indivisible items. Moreover, because of the subjective concept of fairness we define with $v_i$ the subjective utility function of player $i$ that assigns to each possible subset $\bar M$ of items of $M$ a value $v_i(\bar M)$. In the next pages I am going to consider that the utility functions of the players are additive functions.

\subsection{Proportionality and Envy Freeness}
We say that an allocation $\mathcal{A} = (A_1, A_2, \dots, A_n)$ is proportional if for any player $p_i$ holds that $v_i(A_i) \ge \frac{v_i(M)}{n}$. Another criteria, other than proportionality, is Envy freeness that has been introduced in \cite{Foley1967-EF} \cite{VARIAN197463-EF} and we have that an allocation $\mathcal{A} = (A_1, A_2, \dots, A_n)$ is envy-free (EF) if for every couple of players $i,j\in N, \; v_i(A_i)\ge v_i(A_j)$. 

We can notice that when we have indivisible items there is not always an EF allocation or a proportional one, indeed a simple counterexample for both fairness notions is a case in which we have more than one player and only one good. In this case we have that if there are at least two players that value the only item with a value strictly larger than zero than at least one of the two, the player $p_i$ who does not get such item, will not respect the constraint of envy-freeness and proportionality because $v_i(A_i) = v_i(\emptyset) = 0$. For this reason have been introduced some relaxations of the envy-freeness and proportionality criteria: envy freeness up to one good, envy freeness up to any good, maximin share and pairwise maximin share fairness. For all of the following relaxations there exists the approximated version: we say that an allocation is $\alpha$-criteria if $v_i(A_i) \ge \alpha\;\textit{original constraint}$ with $\alpha \in [0,1]$.

\subsection{Envy Freeness up to One Good}
The first introduced relaxation of the envy-free criteria has been the envy-free up to one good criteria. This relaxation has been introduced in \cite{DBLP:LiptonMMS04-EF1} and we have that an allocation $\mathcal{A} = (A_1, A_2, \dots, A_n)$ is envy-free up to one good (EF1) if for every couple of players $i,j\in N$ with $A_j\ne \emptyset, \; \exists g\in A_j: v_i(A_i)\ge v_i(A_j\setminus \{g\})$.

This relaxation of the envy-free criteria has been extensively studied and today are well known two algorithms: round robin and envy cycle elimination that respectively produce EF1 allocation in the case of additive and arbitrary monotonic valuation functions.

\paragraph{Round Robin Algorithm}
Is well known an algorithm to obtain EF1 allocations with $n$ players with additive valuation function and $m$ items, such algorithm is the round robin algorithm.
\begin{algorithm}
\caption{Round Robin Algorithm}\label{alg:round-robin}
$S\gets M$\;
\While{$S \neq \emptyset$}{
    \For{$p_i\in N$}{
        $g \in \argmax_{x\in S} v_i(x)$\; 
        $S \gets S\setminus \{g\}$\;
        $A_i \gets A_i \bigcup \{g\}$\;
    }
}
\end{algorithm}
Is easy to prove that such algorithm produces an EF1 allocation: let's consider a couple of players $i$, $j$ such that $j$ chooses before $i$. As first thing we have to notice that $|A_j| \ge |A_i|$ and that we can have only two cases:  $|A_i| = |A_j| $ or $|A_j| = |A_i| + 1$. Now let's sort the items inside $A_j$ and $A_i$ with respect to the function $v_i$. In both cases we have that the condition of \textit{EF1} must hold since we will have that $v_i(A_{i,k})\ge v_i(A_{j,k+1}) $ 
(where $A_{x,y}$ is the y-th item in the sorted set $A_x$) since the player $i$ surely chooses the $k$-th item before the player $j$ chooses $k+1$ items. So in both cases is enough to use $g = A_{j,0}$ in order to have that the requirement holds. 
As we can see the only difference between the case in which $|A_i| = |A_j| $ and the one in which $|A_j| = |A_i| + 1$ is that in the first one the last item of the sorted set $A_i$, $A_{i, |A_{i}|}$, does not have a related $A_{j, |A_{i}| + 1}$, while in the latter case it has. So also in both cases, by  assigning $g = A_{j,0}$ (ignoring that item) we have that 
$$
    \sum_{k=1}^{|A_j|} v_i(A_{j,k}) \le \sum_{k=0}^{|A_i|} v_i(A_{i,k}) 
$$
If instead $i$ chooses before $j$, player $i$ will always chose before $j$ so he will always have $v_i(A_i)\ge v_i(A_j)$ and so the requirement to be  \textit{EF1} holds also in this case.

\paragraph{Envy Cycle Elimination}
More generally in \cite{DBLP:LiptonMMS04-EF1} has been developed also the Envy Cycle Elimination Algorithm to compute EF1 allocations in the case in which the agents have arbitrary monotonic utilities, so in the case in which 
$$v_i(S) \le v_i(T)\; \forall S\subseteq T\subseteq M\;\forall i\in N$$
This algorithm is based on the definition of envy-graph of a partial allocation $\mathcal{\bar A} = (\bar A_1, \bar A_2, \dots, \bar A_n )$ as a graph where the nodes are the $n$ players and there is a direct edge between $p_i$ and $p_j$ only if $p_i$ envies $p_j$: $v_i(A_i) < v_i(A_j)$. From this definition we can notice that if $p_i$ is a source of the graph, he is not envied by any player, so if we start from a partial EF1 allocation, construct its envy-graph and assign one not allocated item to a player that is a source, the result is still an EF1 allocation since no agent envies the bundle of this player when we remove the last added item because it was a source. It can happen that there is no source in the envy-graph, than if there are edges, there must be a cycle: in order to deal with the cycles is enough to rotate the bundles along the cycle. When we remove a cycle we can notice that, since no other edge will be created because we are not changing the bundles but only changing the owner and each player involved in the bundle swap will increase its utility, the number of edges in the graph will decrease at least by the number of edges in the cycle and this implies the fact that the algorithm terminates.

\begin{algorithm}
\caption{Envy-Cycle Elimination Algorithm}\label{alg:Envy-CycleElimination}
$S\gets M$\;
$\mathcal{\bar A}\gets (\emptyset, \emptyset, \dots,\emptyset)$\;
\While{$S \neq \emptyset$}{
   \While{There is a cycle}{
        remove such cycle;
    } 
    //since there are no cycles there must be a source\;
    allocate item $g\in R$ to a source player\;
}
\end{algorithm}

\subsection{Maxmin Share}
This fairness criteria has been introduced in \cite{DBLP:Budish10-mms} and is a relaxation of the proportionality criteria introduced for the same reason of the EF1 criteria: with indivisible items we cannot always reach proportionality. In order to describe this cirteria we have to introduce the definition of \textit{n-maximin share} of agent $i$ as $\boldsymbol \mu_i = \max_{\mathcal{A} \in \prod_n(M)} \min_{A_j\in \mathcal{A}} v_i(A_j)$. We say that an allocation $\mathcal{A} = (A_1, A_2, \dots, A_n)$ respect the maxmin share criterion if $$\forall i\in N:\; v_i(A_i) \ge \boldsymbol \mu_i$$

\subsection{Pairwise Maximin Share Fairness}
Introduced in \cite{DBLP:/CaragiannisKMP016-EFX-PMMS}, we say that an allocation $\mathcal{A} = (A_1, A_2, \dots, A_n)$ respects the pairwise maximin share (PMMS) criteria if, considering $(B_i, B_j)\in \prod_2(A_i\cup A_j)$,  $\forall i,j\in N$ holds that 
$$
v_i(A_i) \ge \max_{(B_1, B_2)} \min \{v_i(B_1), v_i(B_2)\}
$$

\subsection{Envy Freeness up to Any Good}
Envy Freeness up to any good is a relaxation of the envy-free criteria that is much stronger than the EF1 and has been introduced because this kind of allocation are good approximation to the EF ones.  
This relaxation has been introduced in \cite{DBLP:/CaragiannisKMP016-EFX-PMMS} \cite{DBLP:GourvesMT14-EFX} and we have that an allocation $\mathcal{A} = (A_1, A_2, \dots, A_n)$ is envy-free up to any good (EFX) if for every couple of players $i,j\in N$ with $A_j\ne \emptyset, \; \forall g\in A_j: v_i(A_i)\ge v_i(A_j\setminus \{g\})$.

In the last years this criteria has been extensively studied: 
\begin{itemize}
    \item in 2016 has been given a formal definition,
    \item in 2018 has been shown that with the divide and choose algorithm \cite{DBLP:cut-and-choose-indivisible} we can obtain EFX for two players or $n$ players with identical valuation functions.
    \item in 2019 has been shown that we can build allocations that are EFX and have at least half of the maximum possible Nash Welfare by assigning to the agents only a subset of the items and giving the remaining ones to charity \cite{DBLP:efx-charity}.  
    \item in 2020 has been shown that for $3$ players always exists an EFX allocation \cite{DBLP:3p-efx-existance}.
\end{itemize}

So summarizing till now we have the following results with respect to the number of agents:
\begin{itemize}
    \item $2$ players: the divide and choose algorithm \cite{DBLP:cut-and-choose-indivisible} produces an EFX allocation in polynomial time.
    \item $3$ players: always exists an EFX allocation, but till now we only have a pseudo-polynomial algorithm \cite{DBLP:3p-efx-existance}.
    \item $\ge4$ players: there exists an EFX allocation if we consider only a subset of the entire set of items \cite{DBLP:efx-charity}.
\end{itemize}

Other than this, in \cite{DBLP:BarmanM17-efx-same-ordering} has been shown that when the values of the items have the same ordering for the players than EFX allocation exists and that such EFX allocation is also a $\frac{2}{3}$-MMS allocation. Instead in \cite{DBLP:PlautR17-1/2-EFX-approx} has been proposed an algorithm to obtain $\frac{1}{2}$-EFX allocation when players have sub-additive valuations.
Finally in  \cite{DBLP:MaximumNashWelfareandOtherStoriesAboutEFX} has been proposed a modified version of the round robin algorithm that obtains EFX allocation in the case of additive valuation functions where for each player $p_i$ the values of the items in $M$ have value in the range $[x_i, 2x_i]\; x_i\in R_{>0}$.

\paragraph{Cut and Choose Algorithm}
The cut and choose algorithm introduced, for indivisible items, in \cite{DBLP:cut-and-choose-indivisible} produces EFX allocation for two players with additive valuation functions and is described in \ref{alg:cut-and-choose}. In this algorithm we use the \textit{Leximin++Solution} function that, given the number of partitions, the set of items, and the valuation function of $p_1$ returns, in the case of $2$ players, two set of items that are EFX for $p_1$. 
\begin{algorithm}
\caption{Cut and Choose Algorithm}\label{alg:cut-and-choose}
$(A_1, A_2) \gets$ Leximin++Solution($2, M, v_1$)\;
\eIf{$v_2(A_1) \ge v_2(A_2)$}{
 return ($A_2, A_1$)\;
}
{
 return ($A_1$, $A_2$)\;
}
\end{algorithm}

\subsection{Relations among the fairness criteria}

\begin{figure}[h]
\[\begin{tikzcd}
  &&& {\frac{2}{3}\text{-}PMMS} & {\frac{1}{2}\text{-}EF1} & {\frac{1}{2n-1}\text{-}MMS} \\
  EF & PMMS & EFX & {\frac{4}{7}\text{-}MMS } & {\frac{1}{n}\text{-}MMS} & {\frac{1}{3}\text{-}EF1} \\
  &&& EF1 & {\frac{1}{2}\text{-}PMMS} & {\frac{1}{3n-2}\text{-}MMS}
  \arrow[Rightarrow, from=2-1, to=2-2]
  \arrow[Rightarrow, from=2-3, to=2-4]
  \arrow[Rightarrow, from=2-3, to=1-4]
  \arrow[Rightarrow, from=2-3, to=3-4]
  \arrow[Rightarrow, from=1-4, to=1-5]
  \arrow[Rightarrow, from=1-5, to=1-6]
  \arrow[Rightarrow, from=3-4, to=3-5]
  \arrow[Rightarrow, from=3-4, to=2-5]
  \arrow[Rightarrow, from=3-5, to=3-6]
  \arrow[Rightarrow, from=3-5, to=2-6]
  \arrow[Rightarrow, from=2-2, to=2-3]
\end{tikzcd}
\]
\caption{Relations among the fairness criteria}
\label{fig:relations-fairness-criteria}
\end{figure}
The relations shown in Figure \ref{fig:relations-fairness-criteria} is the summary of several works: in \cite{DBLP:/CaragiannisKMP016-EFX-PMMS} has been shown that $EF\implies PMMS\implies EFX \implies EF1$, instead in \cite{DBLP:ComparingApproximateRelaxationsofEnvy-Freeness} has been shown that in the case of $n\in \{2,3\}$ $EFX\implies \frac{2}{3}\textit{-MMS}$, instead in the case of $n\ge 4$ $EFX\implies \frac{4}{7}\textit{-MMS}$. 

In \cite{DBLP:ComparingApproximateRelaxationsofEnvy-Freeness} has also been analyzed which kind of allocation are implied when we have an $\alpha$-EF1 or $\alpha$-PMMS allocation.
In the case of $\alpha$-EF1 allocation they proved that for any $\alpha\in (0,1]$ an $\alpha$-EF1 allocation is also a $\frac{\alpha}{n-1+\alpha}$-MMS allocation and that $\alpha$-EF1$\implies \frac{\alpha}{1+alpha}$-PMMS for $n\ge 3$.
In the case of $\alpha$-PMMS allocation they proved that for any $\alpha\in (0,1)$ we have that $\alpha\textit{-}PMMS\implies \frac{\alpha}{2-\alpha}\textit{-}EF1$ and that considering also $n\ge 3$ any $\alpha$-PMMS is also an $\frac{\alpha}{2(n-1) -\alpha(n-2)}$-MMS.

\subsection{Thesis Objective}
In this thesis I am going to first show that, starting from the Match\&Freeze algorithm, we can obtain EFX allocation for two players and valuation functions that value each item with one out of three possible values. After this, since we already have an algorithm that, given two players and more general valuation functions, produces EFX allocation, I am going to show how to obtain EFX allocation in polynomial time for three players and valuation functions that value each item with one out of three possible values with some other constraint.

\chapter{Three Values and Two Players}
The Match and Freeze algorithm \ref{alg:matchAndFreeze} is an algorithm introduced in \cite{DBLP:MaximumNashWelfareandOtherStoriesAboutEFX} that computes EFX allocations for $n$ players with additive valuation functions with two values $a, b$ such that $a>b\ge0$. This algorithm proceeds in rounds and keeps two sets: $M$ and $L$ respectively the set of unallocated items and the set of active players. In each round, we assign to each active player one item from the $M$ set, except in the last round in which some player could not receive any item. At each round we decide the item to assign to each player by computing the maximum matching over the bipartite graph where the nodes are the active players in $L$ and the unallocated goods in $M$ and the edges are such that $(i,g)\in E$ if $i\in L, g\in M, v_i(g) = a$. Can happen that some agent are not assigned to any item by the maximum matching, in such a case we assign him a arbitrary available good. This can happen to player $i$ for two reasons: 
\begin{itemize}
    \item there are no items in $M$ valued $a$ by $p_i$
    \item there are items in $M$ valued $a$ by $p_i$, but the maximum matching assigned them to other players.
\end{itemize}
Only the second case is relevant in order to obtain EFX allocations: agent $i$ could value lower its bundle than the one of the players who took the items that he valued $a$ in the last iteration. In order to solve this we freeze for $\floor{a-b}{b}$ rounds the players that took the item valued $a$ by $p_i$ so that he can regain the lost value. In such algorithm we not only freeze the agents that took an item valued $a$ by a player $i$ while $p_i$ took an item valued $b$, we also freeze all the agents that took an item valued $a$ by one of players that are going to be frozen at the end of the round. So we define the frozen set $F$ as $F=\{i\in L | \exists j\in L: v_j(g_i) = a, v_j(g_j) = b\}$ where $g_k$ is the item took in the current round by player $p_k$.
\begin{algorithm}
\caption{Match\&Freeze Algorithm}\label{alg:matchAndFreeze}
$L\gets N$\;
$R\gets M$\;
$l=(1,2,\dots, n)$
\While{$R \neq \emptyset$}{
    Construct the bipartite graph $G=(L\cup R, E)$\;
    Compute the maximum matching on $G$\;
    \For{each matched pair $(i,g)$}{
        Allocate good $g$ to agent $i$\;
        Remove $g$ from $R$\;
    }
    \For{each unmatched active agent $i$ w.r.t. l}{
        Allocate one arbitrary unallocated good $g$ to $i$\;
        Remove $g$ from $R$\;
    }
    Construct the set $F$ of agents that need to freeze\;
    Remove agents of $F$ from $L$ for the next $\floor{a}{b} -1$ rounds\;
    Put agents of $F$ to the end of $l$\;
}
return the resulting allocation $\boldsymbol A$\;
\end{algorithm}

\section{Counter Example for Three Values for The Match\&Freeze Algorithm}
We can see in the next example that the algorithm does not work with three values valuation functions even with only two players. The match and freeze counter example for three values is the following one: let's consider the case in which there are $m=5$ items and two players $p_1$, $p_2$ with the valuations of the items expressed in Table \ref{table:counter-example-match-and-freeze-three-values}. By considering $a = 100$, $b=50$ and $c=1$ the output of the algorithm is $A_1 = \{i_1\}$, $A_2 = \{i_2,i_3, i_4,i_5\}$ where $A_i$ is the sets of items $p_i$ takes. This is not an EFX allocation since $v_1(A_1) = 100 < v_1(A_2\setminus \{i_{5}\}) = 150$. 


\begin{table}[h]
\centering
\begin{tabular}{|l|l|l|l|l|l|}
\hline
      & $i_1$ & $i_2$ & $i_3$ & $i_4$ & $i_5$ \\ \hline
$p_1$ & $a$   & $b$   & $b$   & $b$   & $b$   \\ \hline
$p_2$ & $b$   & $c$   & $c$   & $c$   & $c$   \\ \hline
\end{tabular}
\caption{Counter example for the Match\&Freeze algorithm}
\label{table:counter-example-match-and-freeze-three-values}
\end{table}

\section{Match\&Freeze Modification}
In this section we try to modify the Match and Freeze algorithm in order to obtain an EFX allocation with additive valuation functions, when there are only three possible values for the items, and we have only two players. The first thing we notice from the counter example of the previous section, is that when there are three possible values for the goods, as we freeze player $p_1$, it is possible for her to envy the non-frozen player $p_2$, since $p_2$ has to get $\left \lfloor \frac{b}{c}\right \rfloor -1$ items of value $c$ which have value $b$ for $p_1$. This implies that the value of these items can be $\left \lfloor \frac{b}{c}\right \rfloor b - b$ for $p_1$, a value that is higher than $a$ in case $\left \lfloor \frac{b}{c}\right \rfloor > \left \lfloor \frac{a}{b}\right \rfloor$. This would obviously lead to a non-EFX allocation in the end (from the perspective of player $p_1$).

%%%%%%%%%%%%%%%%%%%%%%%%%%%%%%%%%%%%%%%%%
\subsection{The Modification Approach}
%%%%%%%%%%%%%%%%%%%%%%%%%%%%%%%%%%%%%%%%%

In order to solve this problem, we will apply two modifications to the algorithm:
\begin{enumerate}
    \item We will run the algorithm till the end, and if the produced allocation is not EFX, we will return back to the point where the first player was frozen and exchange the items between the two players (i.e. the items that were assigned to the players at this specific maximum matching)
    \item We will redefine the number of iterations for which a player is frozen in order to adapt it to the three value's case. To do so, we will use a number of iterations that depends both on the remaining items, and on the values obtained by the two players. Specifically, let's consider a point where one of the players will become frozen. Without loss of generality, suppose that $p_1$ takes item $x$ and $p_2$ took item $y$,
the cases in which we freeze player $p_1$ are the following: 
\end{enumerate}


\begin{itemize}
    \item \label{item:freezing-case-take-not-ordered}$v_2(x) = a,\; v_2(y) =  b$ in this case we have that $p_2$ can still take items with value $b$ and $c$. We will unfroze $p_1$ only when $p_2$ has obtained a set of items $F$ while $p_1$ is frozen such that
    \begin{equation}
        a-b\ge v_2(F)\ge a-b-c
        \label{eq:constraint-value-freeze-a-b-c}
    \end{equation}
    The set of items $F$ is obtained by first taking all the items with value $b$ for $p_2$ by respecting that $a-b\ge v_2(F)$, then by taking items with value $c$ for $p_2$ in order to respect the constraint described in equation \ref{eq:constraint-value-freeze-a-b-c}.
    
    Since it is possible that there might not be enough items of value $c$ to reach the aforementioned bound,in that case we will use a different constraint: $ a-b\ge v_2(F)\ge a-2b$. Under this new threshold two possible cases must be considered:
    \begin{itemize}
        \item $\min_{j\in A_1} v_2(j) = b$ 
        \item $\min_{j\in A_1} v_2(j) = c$: in this case we have that at least one item of value $c$ for $p_2$ has been allocated to $p_1$ before she was frozen, so in the same iteration $p_2$ took an item of value at least $b$ for her
    \end{itemize}
    \item $v_2(x) = a,\; v_2(y) =  c$ in this case we have that $p_2$ can only take items of value $c$, so $p_1$ will be frozen for $\left\lfloor \frac{a-c}{c}\right \rfloor =\left \lfloor \frac{a}{c}\right \rfloor - 1$.
    \item $v_l(x) = b,\; v_l(y) =  c$ as in the case, we have that $p_2$ can only take items of value $c$, so $p_1$ will be frozen for $\left \lfloor \frac{b-c}{c}\right \rfloor =\left  \lfloor \frac{b}{c}\right \rfloor - 1$.
\end{itemize}
In order to show that this approach is effective, we have to define the kind of assignment that could lead to a non-EFX allocation, and also prove that by exchanging the items of the players at the problematic point, will result to an EFX allocation.

These modifications to the original algorithm will cause that some assertions made in the original algorithm no longer will hold: the two most important are that 
\begin{itemize}
    \item a player does not always take the items following the order of the values he assign to them as in the case reported above \ref{item:freezing-case-take-not-ordered}
    \item a player could be involved in more than one freeze as we will see in \ref{section:secondproblassignment-2players-3values}
\end{itemize}

%%%%%%%%%%%%%%%%%%%%%%%%%%%%%%%%%%%%%%%%%%%%%%%%%
\subsection{Defining the Problematic Assignments and Resolving the Cases where they do not Occur}
%%%%%%%%%%%%%%%%%%%%%%%%%%%%%%%%%%%%%%%%%%%%%%%%%


We define a \emph{problematic assignment} as an assignment after which a player $p_i$ is frozen, and there are still items that $p_i$ values as $b$. Now let us show that if this kind of assignment does not appear during the execution of the algorithm, the final allocation will be EFX. If we consider that there are no problematic assignments, we can have only the following two cases
\begin{enumerate}
    \item No player will ever be frozen
    \item A frozen player has a value of $c$ for each of the remaining items 
\end{enumerate}

Let set $A_i[j]$ to be the $j$-th item took by player $i$. It is easy to see that in the first case we will have an EFX allocation since  $v_1(A_1[l])\ge v_1(A_2[l])\; \forall l \in \{1,2,\cdots,k\}$ where $k = \min (|A_1|, |A_2|)$ (the same holds for $p_2$).
So we have that if $p_1$ and $p_2$ get the same number of item,s they are envy-free towards each other, otherwise they are EFX since the last item taken by one of the two players will have the smallest value for the other one.

 In the second case we can observe that only one player will be frozen (and only once) during the execution of the algorithm. This is due to the fact that if there are no problematic assignments, when a player $p_i$ is frozen, then then the remaining items have value $c$ for her. This implies that she cannot be frozen again, while the maximum matching procedure guarantees that the algorithm will never freeze the other player as well.  Indeed the only type of assignment that can generate a second freeze needs to have both players with an item of value $b$, as it can be seen in Table \ref{table:bbcc-block}. We proceed by showing that in this case as well, an EFX allocation is produced.
 
Let's consider that at iteration $i$ the algorithm freezes $p_1$, before iteration $i$ we have that $v_1(A_1[l])\ge v_1(A_2[l]) \; \forall l \in \{1,2,\cdots, i-1\}$ and $v_2(A_2[l]) \ge v_2(A_1[l])\;\forall l \in \{1,2,\cdots,i-1\}$. Recall that this is the first time a player is frozen since, as we already mentioned,  only one player can be frozen and only once during the execution of the algorithm.  Let $\lambda$ be the number of items that $p_2$ gets after $p_1$ is unfrozen, $p_2$, something that implies that $p_1$ will take or $\lambda$ or $\lambda-1$ items (at the end we give precedence to the player that has not been frozen).
Finally, let $F$ be the set of items that $p_2$ has taken while $p_1$ was frozen. We proceed in analyzing the different cases:

\begin{enumerate}
    \item $p_1$ took item $x: v_1(x) = a$
    \begin{enumerate}
        \item $p_2$ took item $y:v_2(y) = b$ and $v_2(x) = a$ \label{enumerate:non-problematic-assigment-case-aba}
        %\begin{itemize}
            %\item $p_1$ EFX to $p_2$: we have that the larger amount of iterations that $p_1$ can be frozen is $\lfloor \frac{a-b}{c}\rfloor$ so in this case $v_1(F) \le \lfloor \frac{a-b}{c}\rfloor c \le a-b$. This implies that $v_1(A_1) \ge v_1(A_1^*) + a + (z-1) c$. Instead for what concerns the valuation of player $p_1$ for the items obtained by $p_2$ we can write the following: 
            %\begin{align*}
%v_1(A_2) &\le v_1(A_2^*) + b + v_1(F) + zc\\&\le v_1(A_1^*) + b + a - b + zc = (i-1)a + a + zc      
%\end{align*}    
        %\end{itemize}
        \item $p_2$ took item $y: v_2(y) = c$ and $v_2(x) = a$ \label{enumerate:non-problematic-assigment-case-aca}
        \item $p_2$ took item $y: v_2(y) = c$ and $v_2(x) = b$ \label{enumerate:non-problematic-assigment-case-acb}
    \end{enumerate}
    \item $p_1$ took item $x: v_1(x) = b$
    \begin{enumerate}
        \item $p_2$ took item $y: v_2(y) = c$ and $v_2(x) = b$
        \label{enumerate:non-problematic-assigment-case-bcb}
    \end{enumerate}
\end{enumerate}
%By considering $F$ to be the set of items that $p_2$ takes while $p_1$ is frozen, we have that $v_1(f) \le v_2(f) \; \forall f\in F$. So we 
%We can see that in this case $p_2$ is also EFX towards $p_1$ since $v_2(A_2) \ge (i-1)a + b + v_2(F) + zc= (i-1)a  + a + (z-1)c$ since $v_2(F)\ge a-b-c$ and 
%\begin{align*}
%v_2(A_1) &\le (i-1)a + a + zc = (i-1)a + a + zc      
%\end{align*}
%So also $p_2$ is EFX towards $p_1$.
%\\
%In case $1.b$ instead we have that the the number of iterations in which $p_1$ is frozen is $\lfloor \frac{a-c}{c}\rfloor$ so in this case $v_1(F) = \lfloor \frac{a-c}{c}\rfloor c \le a-c$. So in this case we have that
%v_1(A_1) \ge (i-1)a + a + (z-1) c$, while 
%\begin{align*}
%v_1(A_2) &\le (i-1)a + b + v_1(F) + zc\\&\le (i-1)a + b + a - c + zc = (i-1)a + a + zc      
%\end{align*}
%So we have that $p_1$ is EFX towards $p_2$
These are all the possible cases where $p_1$ can value all the remaining items as $c$, and as we will show, in all of these cases, the produced allocation is always EFX.\footnote{The symmetric cases where player $p_2$ is the one that is frozen, can be resolved under the same arguments.}
\paragraph{Case 1c and 2a}
We begin by observing the set of items that is obtained by $p_2$: $A_2$ will be formed by $A_2^*$ that is the set of items obtained by $p_2$ before that $p_1$ is frozen, item $y$ obtained in the iteration in which $p_1$ is frozen, the set of items $F$ that contains the items obtained by $p_2$ while $p_1$ was frozen and $\lambda$ items with value $c$ obtained from the iteration in which $p_1$ is unfrozen. Similarly, we have that $A_1$ will be formed by $A_1^*$, item $x$ that $p_1$ obtained in the iteration that causes him to be frozen, and at most $\lambda$ items with value $c$ after she he has been unfrozen. We point out again, that it is easy to see that the latter items have value $c$ for both players.

We let $v_2(F) > b-2c$, so we have that, 

\begin{align*}
    v_2(A_2) &= v_2(A_2^*) + v_2(y) + v_2(F) + \lambda c\\
    &= v_2(A_2^*) + c + v_2(F) + \lambda c > v_2(A_2^*) + b-c + \lambda c\\
    \\
    v_2(A_1) &= v_2(A_1^*) + v_2(x) + \lambda c\\
    &=v_2(A_1^*) + b + \lambda c\le v_2(A_2^*) + b + \lambda c
\end{align*} 
where $v_2(A_1^*)\le v_2(A_2^*)$ comes from the fact that this is the first freeze that happens.
So the allocation produced by the algorithm in the case \ref{enumerate:non-problematic-assigment-case-acb} and \ref{enumerate:non-problematic-assigment-case-bcb} is EFX.

\paragraph{Case 1b}
In case  \ref{enumerate:non-problematic-assigment-case-aca} we use the same arguments as in cases \ref{enumerate:non-problematic-assigment-case-acb} and \ref{enumerate:non-problematic-assigment-case-bcb} described above, with the only difference that $v_2(F) > a-2c$. In particular,
\begin{align*}
    v_2(A_2) &= v_2(A_2^*) + v_2(y) + v_2(F) + zc\\
    &= v_2(A_2^*) + c + v_2(F) + zc > v_2(A_2^*) + a-c + zc\\\\
    v_2(A_1) &= v_2(A_1^*) + v_2(x) + zc\\
    &=v_2(A_1^*) + a + zc\le v_2(A_2^*) + a + zc
\end{align*}
Thus, we end up with an EFX allocation.

\paragraph{Case 1a}
In case \ref{enumerate:non-problematic-assigment-case-aba} by considering the constraint defined in equation \ref{eq:constraint-value-freeze-a-b-c} we have that $a-b\ge v_2(F)\ge a-b-c$. After $p_1$ is unfrozen, we can have that there are still items valued as $b$ from the perspective of $p_2$. In this case we have that the items obtained after the point where $p_1$ is unfrozen, have to be divided in $\lambda_{b, 1}$, $\lambda_{c, 1}$, $\lambda_{b, 2}$ and  $\lambda_{c, 2}$ as the items that have value $b$ and $c$ for $p_2$, and are obtained by players $p_1$ and $p_2$ respectively.
 So we have the following considering that the non-frozen player has the precedence in the last iteration
\begin{align*}
\lambda_{b,1} = \begin{cases} 
\lambda_{b,2} &\implies \lambda_{c,1} = \begin{cases} 
\lambda_{c,2}\\
\lambda_{c,2} -1
\end{cases}   \\
\lambda_{b,2} -1 &\implies \lambda_{c,1} = \begin{cases} 
\lambda_{c,2}\\
\lambda_{c,2} +1
\end{cases}   
\end{cases}
\end{align*}
The worst case to consider is when $\lambda_{b,1} = \lambda_{b,2} = \lambda_b$ and $\lambda_{c,1} = \lambda_{c,2} = \lambda_c$. In this case we have that $A_1$ is formed by $A_1^*$, item $x$ obtained in the iteration that causes $p_1$ to be frozen, $z_b$ and $z_c$ as the items obtained after that he has been unfrozen with value $b$ and $c$ for $p_2$. Instead $A_2$ is formed by $A_2^*$, item $y$ obtained in the iteration that causes $p_1$ to be frozen, $F$ the set of items obtained while $p_1$ was frozen and $z_b$ and $z_c$ as the items obtained after that he has been unfrozen respectively with value $b$ and $c$ for $p_2$. So we can write that 
\begin{align*}
    v_2(A_2) &= v_2(A_2^*) + v_2(y) + v_2(F) + z_b b + z_cc\\
    &= v_2(A_2^*) + b + v_2(F) + z_bb +  z_cc > v_2(A_2^*) + a-c + z_b b + z_cc\\\\
    v_2(A_1) &= v_2(A_1^*) + v_2(x) + z_bb + z_cc \\
    &=v_2(A_1^*) + a + z_bb + z_cc\le v_2(A_2^*) + a + z_bb + z_cc
\end{align*}
In the above equations I have considered that there are enough items with value $c$ for $p_2$ to respect the constraint defined in \ref{eq:constraint-value-freeze-a-b-c}, but if there are not enough of this items we have that $v_2(F) \ge a-2b$ and the two following possible cases:
\begin{itemize}
        \item $\min_{i\in A_1} v_2(i) = b$: in this case the value of $A_2$ for $p_2$ becomes:
        \begin{align*}
            v_2(A_2) &= v_2(A_2^*) + v_2(y) + v_2(F) + z_b b \\
            &= v_2(A_2^*) + b + v_2(F) + z_bb > v_2(A_2^*) + a-b + z_b b
        \end{align*}
       that still is an EFX allocation since we have that from $v_2(A_1) \le v_2(A_2^*) + a + z_bb$ we have to remove $b$.
        \item $\min_{i\in A_1} v_2(i) = c$: in this case we would have that this item $i_c$ with value $c$ for $p_2$ has been taken before that $p_1$ has been frozen, so in the same iteration $p_2$ took an item $i_b$ with value at least $b$ for $p_2$. So we can write the precedent equations as:
        \begin{align*}
            v_2(A_2) &= v_2(A_2^*\setminus{\{i_b\}}) + v_2(i_b) + v_2(y) + v_2(F) + z_b b \\
            &= v_2(A_2^*\setminus{\{i_b\}}) + b + b + v_2(F) + z_bb  > v_2(A_2^*) + a + z_b b\\\\
            v_2(A_1) &= v_2(A_1^*\setminus{\{i_c\}}) + v_2(i_c)+ v_2(x) + z_bb  \\
            &=v_2(A_1^*\setminus{\{i_c\}}) + c  + a + z_bb \le v_2(A_2^*) + c + a + z_bb 
        \end{align*}
\end{itemize}
So also in case \ref{enumerate:non-problematic-assigment-case-aba} the algorithm produces EFX allocations. For the next cases I will not consider the case in which we do not have enough items with value $c$ for the non frozen player because as shown in the precedent part, we still have that the difference between the value for $p_2$ of $A_1$ and $A_2$ differs for at most the value of the item with lower value for $p_2$ in $A_1$ after that $p_1$ is unfrozen.


\paragraph{}
Now what remains to  be shown is that the produced allocation by the algorithms is EFX, even if there exists a problematic assignment. In particular, we want to show that if after a problematic assignment the first run of the algorithm produces an allocation that is not EFX, then we can go back, change the assignment at thus point, run the algorithm from this point and onward, and obtain an EFX allocation. 
The possible problematic assignments are shown in Table \ref{table:problematic-assignments-2-players}:
\begin{table}[h]
    \footnotesize
 
        \centering
       \begin{tabular}{|l|l|l|}
            \hline
            $p_1$ & $a$ & $b$ \\ \hline
            $p_2$ & $b$ & $c$ \\ \hline
        \end{tabular}
        \;\;\;\;\;\;\;\;\;\;\;\;
        \centering
        \begin{tabular}{|l|l|l|}
            \hline
            $p_1$ & $a$ & $b$ \\ \hline
            $p_2$ & $a$ & $b$ \\ \hline
        \end{tabular}
        
        \caption{Problematic Assignments}
        \label{table:problematic-assignments-2-players}
    \end{table}
    
\subsection{First Problematic Assignment}
\begin{table}[]
    \centering
    \begin{tabular}{|l|l|l||l|}
    \hline
    $p_1$ & \textbf{a} & $b$ & $b$        \\ \hline
    $p_2$ & $b$        & $c$ & \textbf{b} \\ \hline
    \end{tabular}
    \label{table:first-problem-association-show-no-double-freeze-first}
    \caption{Assignment that shows that with the first problematic assignment we cannot have two frozen players or one player frozen twice}
\end{table}

Let's start by showing that like in the precedent cases without problematic assignment, we have that only one player per run can be frozen. As first thing we can notice that in order to have a second freeze we must have that there is the assignment shown in Table \ref{table:bbcc-block}, so we need to have an item valued by both players $b$ after that a player has been frozen; but if this happens, than we had not followed the maximum matching in the first execution of the algorithm since we could have assigned the items as shown in Table \ref{table:first-problem-association-show-no-double-freeze-first} in bold. Indeed we have that $a+b > b + b$ and $a+b > a+c$.

Let's consider the first problematic assignment and that $a + c \ge b + b$: in this case by running the algorithm we obtain the following valuations for the two players
\begin{align*}
    v_1(A_1) &= v_1(A_1^*) + a + w_b b + w_c c\\
    v_2(A_1) &= v_2(A_2^*) + c + v_2(F_1)+ w_b c + w_c c
\end{align*}
where $A_i^*$ is the set of items obtained in the iterations before the problematic assignment by $p_i$, $w_b$ and $w_c$ are respectively the number of items with value $b$ and $c$ obtained by $p_1$ after that he has been unfrozen and $F_1$ is the set of items obtained by $p_2$ while $p_1$ was frozen. If we consider that after that the frozen player is unfrozen, both player get the same number of items in order to have an EFX allocation we must respect the following two constraints since we assume that in the last iteration the non frozen player has the precedence. 
\begin{align}
    v_1(A_1) &\ge v_1(A_2)\label{eq:condition-1-a1>a2-ch1-first-block-ac-assignment}\\
    v_2(A_2) &\ge v_2(A_1) - c\label{eq:condition-1-a2>a1-c-ch1-first-block-ac-assignment}
\end{align}
In this case we have that the set of items $A_1$ will contain $A_1^*$, the item $x$ such that $v_1(x) = a$ and $v_2(x) = b$ that has been obtained by $p_1$ in the iteration that causes $p_1$ to be frozen, $w_b$ and $w_c$ items obtained by $p_1$ after that has been unfrozen that have value $b$ and $c$ respectively for $p_1$. Instead $A_2$ will contain $A_2^*$, the item $y$ such that $v_1(y) = b$ and $v_2(y) = c$ that has been obtained by $p_2$ in the iteration that causes $p_1$ to be frozen, $w_b$ and $w_c$ items obtained by $p_1$ after that has been unfrozen that have value $b$ and $c$ respectively for $p_1$.
In order to respect the condition in \ref{eq:condition-1-a1>a2-ch1-first-block-ac-assignment} we must have that:
\begin{align*}
    v_1(A_1) &\ge v_1(A_2)\\
    v_1(A_1^*) + v_1(x) + w_b b + w_c c &\ge  v_1(A_2^*) + v_1(y) + v_1(F_1)+ w_b b + w_c c \\
    a  &\ge  b + v_1(F_1)
\end{align*}
where we are considering $v_1(A_1^*) = v_1(A_2^*)$ cause this is the worst possible case since we always have that $v_1(A_1^*) \ge v_1(A_2^*)$ and this is due to the fact that these set of items are referred to the iterations before that a player has been frozen, so at each of these iterations we have that the item obtained by player $1$ has a larger or equal value to the item obtained by player $2$ for player $1$ (the same holds for $p_2$).
Instead in order to have that the condition in equation \ref{eq:condition-1-a2>a1-c-ch1-first-block-ac-assignment} we must have that
\begin{align*}
    v_2(A_2) &\ge v_2(A_1) -c\\
    v_2(A_2^*) + c + v_2(F_1)+ w_b c + w_c c &\ge v_2(A_1^*) + b + w_b c + w_c c - c\\
    c  + v_2(F_1)&\ge  b - c 
\end{align*}
where we can notice that the items relative to the counters $w_b$ and $w_c$ have value $c$ for $p_2$ since he after the freeze can only take items with that value.
This condition is always true since $ c  + v_2(F_1) =\left \lfloor \frac{b}{c}\right\rfloor c > b-c$. So the only possible problem is the first condition: $a \ge  b + v_1(F_1)$. Since in $F_1$ there are the items taken by $p_2$ while $p_1$ is frozen, there can be items valued $b$ by $p_1$; let's consider that we first assign to $p_2$ in $F$ the items valued $b$ by $p_1$ that have not yet assigned before the problematic assignment. In this case we can write
\begin{align*}
    v_1(F_1) = kb + \left\lfloor \frac{b-c-kc}{c}\right\rfloor c = kb + (\left\lfloor \frac{b}{c}\right\rfloor -1-k)c
\end{align*}
where $k$ is the number of items valued $b$ by $p_1$ that are in $F_1$ and $\lfloor \frac{b-c-kc}{c}\rfloor$ are the remaining items valued $c$ by $p_1$ that $p_2$ needs to reach $v_2(F_1)  = \lfloor \frac{b-c}{b}\rfloor c$. So we can write the condition in equation \ref{eq:condition-1-a1>a2-ch1-first-block-ac-assignment} as $a \ge b +kb + (\lfloor \frac{b}{c}\rfloor -1-k)c $, that is equivalent to 
\begin{align}
    k \le \frac{a + c - b - \lfloor \frac{b}{c}\rfloor c }{b-c}
    \label{eq:condition-1-a1>a2-ch1-first-block-ac-assignment-on-k}
\end{align}

If we change the problematic assignment we have that we will break the maximum matching rule and we will freeze $p_2$ rather than $p_1$. In this case we are considering $w_b$ and $w_c$ as the number of items with value $b$ and $c$ for $p_1$ respectively that $p_1$ take after that $p_2$ has been unfrozen, so we have that the valuation for the two players will be 
\begin{align*}
    v_1(A_1) &= v_1(A_1^*) + b + v_1(F_2) + w_b b + w_c c\\
    v_2(A_1) &= v_2(A_2^*) + b + w_b c + w_c c
\end{align*}
The conditions in order to have an EFX allocation now become the following ones, since in this case $p_1$ will have the precedence in the last iteration: 
\begin{align}
    v_1(A_1) &\ge v_1(A_2) - c\label{eq:condition-1-a1>a2-ch1-first-block-bb-assignment}\\
    v_2(A_2) &\ge v_2(A_1)\label{eq:condition-1-a2>a1-c-ch1-first-block-bb-assignment}
\end{align}
In this case we have that $A_1$ will contain $A_1^*$, the item $y$ obtained in the iteration in which we freeze $p_2$ such that $v_1(y) =  b$ and $v_2(y) = c$, $F_2$ as the set of items obtained by $p_1$ while $p_2$ is frozen and $w_b$ and $w_c$ items obtained by $p_1$ after that $p_2$ has been unfrozen that have value $b$ and $c$ respectively for $p_1$. Instead $A_2$ will contain $A_2^*$, the item $x$ obtained in the iteration in which we freeze $p_2$ such that $v_1(x) =  a$ and $v_2(x) = b$ and $w_b$ and $w_c$ items obtained by $p_2$ after that has been unfrozen that have value $b$ and $c$ respectively for $p_1$. So the first condition is equivalent to 
\begin{align*}
    v_1(A_1) &\ge v_1(A_2) -c\\
    v_1(A_1^*) + v_1(y) + v_1(F_2)+ w_b b + w_c c &\ge v_2(A_1^*) + v_1(x) + w_b b + w_c b - c\\
    b  + v_1(F_2)&\ge  a - c 
\end{align*}
That is always true since $ v_1(F_2) \ge  a-b-c$ as defined in equation \ref{eq:constraint-value-freeze-a-b-c}. So now we have to check the condition for which $p_2$ is EFX towards $p_1$.
\begin{align*}
    v_2(A_2) &\ge v_2(A_1)\\
    v_2(A_2^*) + v_2(x) + w_b c + w_c c &\ge  v_2(A_1^*) + v_2(y) + v_2(F_2)+ w_b c + w_c c\\
    b  &\ge  c + v_2(F_2)
\end{align*}
We can notice that $v_2(F_2) = kc + \lfloor \frac{a-b-kb}{c} \rfloor c$ since, as said before, $k$ is the minimum number of items that are surely present with value $b$ for $p_1$ after the problematic assignment, so the other values that are used to reach the constraint $v_1(F_2) \ge a-b-c$ are in the worst case items with value $c$ for $p_1$. We can also notice that all these items have value $c$ for $p_2$ since he already took an item with value $c$, so there are no other items with value $b$ or $a$. So we have that the condition in equation \ref{eq:condition-1-a2>a1-c-ch1-first-block-bb-assignment} becomes
\begin{align*}
    v_2(A_2) &\ge v_2(A_1)\\
     b  &\ge  c + v_2(F_2)\\
      b  &\ge  c +  kc + \left\lfloor \frac{a-b-kb}{c} \right\rfloor c\\
      b  &\ge  c +  kc + a-b-kb\\
      \frac{a + c -2b}{b-c}  &\le   k
\end{align*}
So the second condition in order to have an EFX allocation by changing the problematic assignment is 
\begin{equation}
    k \ge \frac{a + c -2b}{b-c}
    \label{eq:condition-1-a2>a1-ch1-first-block-bb-assignment-on-k}
\end{equation}
As we can see, equation \ref{eq:condition-1-a1>a2-ch1-first-block-ac-assignment-on-k} and \ref{eq:condition-1-a2>a1-ch1-first-block-bb-assignment-on-k} are complementary on integer values of $k$. So if the algorithm produces a non EFX allocation, by changing the assignment we surely obtain an EFX allocation in the first problematic assignment of table \ref{table:problematic-assignments-2-players}. 

\subsection{Second Problematic Assignment}
\label{section:secondproblassignment-2players-3values}
In order to show that also in this case we obtain an EFX allocation, I will first consider the case of identical valuation functions since is easier to deal with and than I will show that we can reduce a general case to this one. This reduction will come because I will show that in the case of non identical valuation function, by freezing one of the two players, we can obtain that the value for the frozen player of the set obtained by the non frozen one is lower or equal than the value obtainable in the case of identical valuation functions.  We can easily note that with the second problematic assignment we could have two freezes: one caused by this problematic block and one caused than by the block shown in Table \ref{table:bbcc-block}.
\begin{table}[h]
        \centering
        \begin{tabular}{|l|l|l|}
            \hline
            $p_1$ & $b$ & $c$ \\ \hline
            $p_2$ & $b$ & $c$ \\ \hline
        \end{tabular}
        
        \caption{Second block that can cause a freeze when we have the second problematic block}
        \label{table:bbcc-block}
    \end{table}
In this part I am going to use $F_{i,j}$ in order to describe the set of items obtained by the player who took item with value $j$ while the other player $i$ is frozen and also
\begin{align*}
    &k = a\mod b\\
    &l = b \mod c\\
    &w = k \mod c
\end{align*}
\paragraph{Identical Valuation Functions}
Let's consider the case in which we have identical valuation functions and we have that we freeze first $p_1$ and then $p_2$. In this case we can ignore the items that the players took in the iterations in which no player is frozen and the ones that do not lead to freeze one player because in each iteration each player has the same value for both the items assigned. So by considering only the iterations in which a player is frozen and the two that lead to freeze one player we can consider the valuations of each player set as 
\begin{align*}
    v(A_1) & = a + c + v(F_{2,c})\\
    v(A_2) & = b + v(F_{1,b}) + b
\end{align*}
where $v(\cdot) = v_1(\cdot) = v_2(\cdot)$. These values comes from the fact that in $A_1$ we have the item valued as $a$ that leads to freeze $p_1$, $c$ is the value of the item obtained in the iteration in which we freeze $p_2$ and $F_{2,c}$ is the set of items obtained when $p_2$ is frozen since $p_1$ took the item with value $c$.
Instead in $A_2$ we have the item with value $b$ obtained in the iteration in which $p_1$ is frozen,
$F_{1,b}$ that is the set of items obtained by $p_2$ while $p_1$ is frozen since $p_2$ took the item with value $b$ and $b$ that is the value obtained by $p_2$ before being frozen.
We can write that 
\begin{align}
    v(F_{2,c}) + c &= \left\lfloor \frac{b}{c}\right\rfloor c = b - (b\% c) = b-l \label{v(F2c)+c-identical-valuation-functions-second-block}\\
    v(F_{1,b}) + b &= \left\lfloor \frac{a}{b}\right\rfloor b = a - (a\% b) = a - k  &\textit{ if } \:k = a\% b < c \label{v(F1b)+b-amodb<c-identical-valuation-functions-second-block}\\
    v(F_{1,b}) + b &=\left\lfloor \frac{a}{b}\right\rfloor b + \left\lfloor \frac{a - \left\lfloor \frac{a}{b}\right\rfloor b }{c}\right \rfloor c \label{v(F1b)+b-amodb>c-identical-valuation-functions-second-block} \\&= a-k + \left\lfloor \frac{a- a + k }{c} \right\rfloor c = a - k + \left\lfloor \frac{k}{c}\right\rfloor c \\&= a - k + k - k\% c  = a - w &\textit{ if } \:k = a\% b \ge c
\end{align}
where in equation \ref{v(F1b)+b-amodb<c-identical-valuation-functions-second-block} we are considering that, since $a\% b < c$ than by taking $\left\lfloor\frac{a}{b}\right\rfloor$ items with value $b$ we reach the bound $a-b-c$, while in equation \ref{v(F1b)+b-amodb<c-identical-valuation-functions-second-block} we need to take some items with value $c$ in order to reach it. We can also notice that since we are considering that there is a second freeze, there is a number of items with value $b$ greater or equal to $\left\lfloor\frac{a}{b}\right\rfloor$. 
\begin{itemize}
    \item if $k < c$ we have that
    \begin{align*}
        &v(A_1) = a + c + v(F_{2,c}) && v(A_2) = b + v(F_{1,b}) + b\\
        &v(A_1) = a + b-l && v(A_2) = a - k + b
    \end{align*}
    In the case considered above both players get the same number of items after that the last player has been unfrozen, in this case we can see that the allocation is EFX: 
    \begin{align*}
        &v(A_1) \ge v(A_2) -c & -l \ge - k - c\\
        &v(A_2) \ge v(A_1) -c & -k \ge -l -c
    \end{align*}
    that is true since $k<c$ and $l<c$.
    Instead in the case in which we have an item that is taken by a player and not by the other, by assigning it correctly, we can obtain an EFX allocation: if we assign that item to $p_1$ we have that the condition to have an EFX allocation is
    \begin{align*}
        &v(A_1) \ge v(A_2) -c & -l \ge - k - c\\
        &v(A_2) \ge v(A_1) & -k \ge -l 
    \end{align*}
    that is equivalent to have only the condition $k\le l$.
    While if we assign it to $p_2$ the condition to have an EFX allocation is
    \begin{align*}
        &v(A_1) \ge v(A_2) & -l \ge - k\\
        &v(A_2) \ge v(A_1)-c & -k \ge -l -c
    \end{align*}
    that is equivalent to have the condition $l\le k$.
    
    \item if $k \ge c$ we have that
    \begin{align*}
        &v(A_1) = a + c + v(F_{2,c}) && v(A_2) = b + v(F_{1,b}) + b\\
        &v(A_1) = a + b-l && v(A_2) = a - w + b
    \end{align*}
    In the case considered above both players get the same number of items after that the last player has been unfrozen, in this case we can see that the allocation is EFX: 
    \begin{align*}
        &v(A_1) \ge v(A_2) -c & -l \ge - w - c\\
        &v(A_2) \ge v(A_1) -c & -w \ge -l -c
    \end{align*}
    that is true since $w<c$ and $l<c$
    Instead in the case in which we have an item that is taken by a player and not by the other, by assigning it correctly, we can obtain an EFX allocation: if we assign that item to $p_1$ we have that the condition to have an EFX allocation is
    \begin{align*}
        &v(A_1) \ge v(A_2) -c & -l \ge - w - c\\
        &v(A_2) \ge v(A_1) & -w \ge -l 
    \end{align*}
    that is equivalent to have only the condition $w\le l$.
    While if we assign it to $p_2$ the condition to have an EFX allocation is
    \begin{align*}
        &v(A_1) \ge v(A_2) & -l \ge - w\\
        &v(A_2) \ge v(A_1)-c & -w \ge -l -c
    \end{align*}
    that is equivalent to have the condition $l\le w$.
\end{itemize}

\paragraph{Non Identical Valuation Functions} 
In the precedent paragraph I have shown that with identical valuation functions, also in the case of two freezes, we still can obtain an EFX allocation by correctly assigning the last item. In this section I am going to show that this holds also for non identical functions.
I am going to show that this holds by showing that we have one of the two following conditions

\begin{itemize}
    \item if we first freeze $p_1$ and than $p_2$ we have that $v_1(F_{1,b})\le v_2(F_{1,b})$
    \item if we first freeze $p_2$ and than $p_1$ we have that $v_2(F_{2,b})\le v_1(F_{2,b})$
\end{itemize}
Let's consider without loss of generality that the first condition holds, than after that $p_1$ is unfrozen, he will not envy $p_2$ since $a-b\ge v_2(F_{1,b}) \ge v_1(F_{1,b})$. Than in the iterations before the freeze of $p_1$  and the ones between the one in which we unfroze $p_1$ and the one in which we freeze $p_2$ we have that since no one is frozen, the value of the items obtained by $p_1$ is higher or equal to the value of the items taken by $p_2$ for $p_1$ (so in the worst case the two values will be the same). Than we have the second freeze, after which all the items are valued $c$ by both players, so as we can see the worst case is the one in which we have identical valuation for $p_1$, this holds also for $p_2$ cause in the iterations before that $p_1$ is frozen and in the ones between the one in which we unfreeze $p_1$ and the one in which we freeze $p_2$, the items obtained by $p_2$ have an higher or equal value to the ones obtained by $p_1$ for $p_2$, and after the second freeze, all the items have value $c$ for both players. So also for $p_2$ the worst case in which we have the two freezes is the one with the identical valuation functions.
We can say the same things inverting $p_2$ and $p_1$ if the second condition holds. So we can reduce this case to the identical values functions case.
\\

Let's consider the case in which we freeze first $p_1$, then $p_2$ will take $F_{1,b}$ items while $p_1$ is frozen. Of these items we can have three types of items: 
\begin{enumerate}
    \item item valued by both players as $b$ \label{item:item-valued-bb}%y
    \item item valued by player $p_1$ as $c$ and by $p_2$ as $b$ \label{item:item-valued-cb} %v
    \item item valued by player $p_1$ as $b$ and by $p_2$ as $c$ \label{item:item-valued-bc}%s
\end{enumerate}
The same happens when we have that we freeze $p_2$ and than $p_1$. Let's consider that in both cases we use the same number of items of type \ref{item:item-valued-bb}: $t_{b,b}$, that of type \ref{item:item-valued-cb} we have $t_{c,b}$ items and of type \ref{item:item-valued-bc} we have $t_{b,c}$ items. So we have that in $t_{**}$ the first letter in the subscript is the value for $p_1$, while the second one is the value for $p_2$.
When freezing $p_1$ we can write that 
\begin{align*}
    v_1(F_{1,b}) &= t_{bb}b + t_{cb}c + \left\lfloor \frac{a-b-yb-vb}{c}\right\rfloor b \\
                & = t_{bb}b + t_{cb}c + \hat{t}_{bc} b\\
    v_2(F_{1,b}) &= t_{bb}b + t_{cb}b + \left\lfloor \frac{a-b-yb-vb}{c}\right\rfloor c \\
                & = t_{bb}b + t_{cb}b + \hat{t}_{bc} c \le a-b
\end{align*}
where $\hat{t}_{b,c}\le t_{b,c}$ is the number of items of type $3$ that $p_2$ needs to achieve the constraint $v_2(F_{1,b})\ge a-b-c$. In this case in order to have $v_1(F_{1,b}) \le v_2(F_{1,b})$ we must have that
\begin{align*}
    v_1(F_{1,b}) &\le v_2(F_{1,b})\\
    t_{bb}b + t_{cb}c + \hat{t}_{bc} b &\le t_{bb}b + t_{cb}b + \hat t_{bc} c\\
    \hat t_{bc}(b-c) &\le t_{cb}(b-c)
\end{align*}
that is surely true when $s\le v$ since $b> c$.\\
Instead if we freeze $p_2$ we have that
\begin{align*}
    v_1(F_{2,b}) &= t_{bb}b + t_{bc}b + \left\lfloor \frac{a-b-yb-sb}{c}\right\rfloor c \\
                & = t_{bb}b + t_{bc}b + \hat t_{cb} c\\
    v_2(F_{2,b}) &= t_{bb}b + t_{bc}c + \left\lfloor \frac{a-b-yb-sb}{c}\right\rfloor b \\
                & = t_{bb}b + t_{bc}c + \hat t_{cb} b \le a-b
\end{align*}
where $ \hat t_{c,b}\le t_{c,b}$ is the number of items of type $2$ that $p_1$ needs to achieve the constraint $v_1(F_{2,b})\ge a-b-c$. In this case in order to have $v_2(F_{2,b}) \le v_1(F_{2,b})$ we must have that
\begin{align*}
    v_2(F_{2,b}) &\le v_1(F_{2,b})\\
    t_{bb}b + t_{bc}c +  \hat t_{cb} b &\le t_{bb}b + t_{bc}b +  \hat t_{c,b} c\\
    \hat t_{cb}(b-c) &\le t_{bc}(b-c)
\end{align*}
that is surely true when $t_{cb}\le t_{bc}$ since $b> c$.\\
So depending on $t_{cb}$ and $t_{bc}$ we can choose the player to freeze first and in the worst case we will obtain that the value of the set of items obtained by the unfrozen player for the frozen one is equal to the value for the unfrozen player.
This worst case is leads to the same results obtained when we have identical valuation functions, so also if we have two freezes, than we still can obtain an EFX allocation by correctly assign the last item and by going back to the first freeze if we do not obtain an EFX allocation.


\paragraph{No double freeze} In the case in which we have that there is the second problematic assignment and no second freeze, we can see that we still obtain an EFX allocation by considering that in the precedent paragraph we have shown that in a general case one of the following two cases is true
\begin{itemize}
    \item if we first freeze $p_1$ and than $p_2$ we have that $v_1(F_{1,b})\le v_2(F_{1,b})$
    \item if we first freeze $p_2$ and than $p_1$ we have that $v_2(F_{2,b})\le v_1(F_{2,b})$
\end{itemize}
Let's consider that the first condition is true, than we have that when $p_1$ is unfrozen he will not envy $p_2$. If after that $p_1$ is unfrozen there are no freezes, than $p_1$ will take items with value higher or equal to the one obtained by $p_2$ for $p_1$, so at the end he will be EFX towards $p_2$ because at most $p_2$ will take the last item that will have the lower value for $p_1$ among the ones obtained by $p_2$. Instead for what concerns $p_2$, after that $p_1$ is unfrozen, he will have that $v_2(A_2) \ge v_2(A_1) - c$ because of the constraint described in equation \ref{eq:constraint-value-freeze-a-b-c}. In the iterations after that $p_1$ is unfrozen, for $p_2$ holds the same thing that holds for $p_1$ with the only exception that $p_2$ has precedence in the last iteration, so also $p_2$ will be EFX towards $p_1$. So also in this case we obtain an EFX allocation.

\subsection{Algorithm Cost}
With respect to the original algorithm that worked on two values valuation functions, we have that the number of steps the algorithm does increases because of the fact that in case of non EFX allocation we have to go back to the iteration with one of the two problematic assignments and run again the algorithm. We could have also that the first assignment of the algorithm is a problematic assignment and this would mean that we could have to run the double of the steps required by the original algorithm. 

\subsection{Unify Non Problematic Assignment Proofs}
In this section I am going to give a unified proof that in the case of non problematic assignments we can obtain EFX allocation by following only the new rules for allocate items while a player is frozen.  Let's consider the fact that $A_1^*$ and $A_2^*$ are respectively the set of items obtained by $p_1$ and $p_2$ before the freeze and that we freeze player $p_1$ in an iteration in which he took an item $x$ and $p_2$ took an item $y$, moreover let's consider that $p_2$ took a set of items $F$ while $p_1$ is frozen. It's easy to see that since all the remaining items for the frozen player $p_1$ have value $c$ for him, than he will be EFX towards $p_2$ in the end, because the items taken by $p_2$ will have higher or equal value to $c$ and so $v_2(F) + v_2(y) \ge v_1(F) + v_1(y)$ and, since $v_1(x) \ge v_2(x) \ge v_2(F) + v_2(y)$ we have that $v_1(x) \ge v_1(y) + v_1(F)$.
For what concerns $p_2$ we can write for all the non problematic cases the following equation because of the definition of the items assigned while a player is frozen:
\begin{equation*}
    v_2(x) - v_2(y) \ge v_2(F) \ge v_2(x) - v_2(y) - v_2(l_1)
\end{equation*}
where $l_1$ is the item with lower value for $p_2$ in the set $A_1$ at the end of the algorithm. By defining $\hat A_i$ as the set of items obtained by player $i$ after that $p_1$ has been unfrozen, we can notice that $v_2(\hat A_2) \ge v_2(\hat A_1)$ since there are no other freezes and because the last iteration $p_2$ has precedence.
So, because of we have that $A_2 = A_2^* \cup \{y\} \cup F \cup \hat A_2 $ and that $A_1 = A_1^* \cup \{x\} \cup \hat A_1 $ we can write 
\begin{align*}
    v_2(A_2) &=v_2(A_2^*) +  v_2(y) + v_2(F) + v_2(\hat A_2)\\
    & \ge  v_2(A_2^*) +  v_2(x) - v_2(l_1) + v_2(\hat A_2)\\\\
    v_2(A_1)& = v_2(A_1^*) + v_2(x) + v_2(\hat A_1)\\
    &\le v_2(A_2^*) + v_2(x) + v_2(\hat A_2)
\end{align*}
That clearly means that the allocation is EFX also for $p_2$ for the definition of $l_1$.
In the precedent proof I have considered that in the case \ref{enumerate:non-problematic-assigment-case-aba} there are enough items with value $c$ for $p_2$ to reach the constraint described in \ref{eq:constraint-value-freeze-a-b-c}, but if there are not enough of this items we have that $v_2(F) \ge a-2b$ and the two following possible cases:
\begin{itemize}
        \item $\min_{i\in A_1} v_2(i) = b$: in this case the value of $A_2$ for $p_2$ becomes:
        \begin{align*}
            v_2(A_2) &= v_2(A_2^*) + v_2(y) + v_2(F) + v_2(\hat A_2)\\
            &= v_2(A_2^*) + b + v_2(F) + v_2(\hat A_2) > v_2(A_2^*) + a-b + v_2(\hat A_2)
        \end{align*}
       that still is an EFX allocation since we have that from $v_2(A_1) \le v_2(A_2^*) + a + v_2(\hat A_2)$ we have to remove an item valued $b$ for the definition of EFX.
        \item $\min_{i\in A_1} v_2(i) = c$: in this case we would have that this item $i_c$ with value $c$ for $p_2$ has been taken before that $p_1$ has been frozen, because we are in the case in which the non frozen player has not enough items valued $c$ to reach the constraint defined in \ref{eq:constraint-value-freeze-a-b-c}, so such items is surely not assigned after that $p_1$ is unfrozen. By considering that in the same iteration in which $p_1$ took the item $i_c$ $p_2$ took an item $i_b$ with value at least $b$ for $p_2$ (because this iteration is before the freeze of $p_1$ so there are still items valued at least $b$ by $p_2$). So we can write the precedent equations as:
        \begin{align*}
            v_2(A_2) &= v_2(A_2^*\setminus{\{i_b\}}) + v_2(i_b) + v_2(y) + v_2(F) + v_2(\hat A_2) \\
            &= v_2(A_2^*\setminus{\{i_b\}}) + b + b + v_2(F) + v_2(\hat A_2)  > v_2(A_2^*) + a + v_2(\hat A_2)\\\\
            v_2(A_1) &= v_2(A_1^*\setminus{\{i_c\}}) + v_2(i_c)+ v_2(x) + v_2(\hat A_1)  \\
            &=v_2(A_1^*\setminus{\{i_c\}}) + c  + a + v_2(\hat A_1) \le v_2(A_2^*) + c + a + v_2(\hat A_2) 
        \end{align*}
        Because $A_2 = A_2^*\setminus{\{i_b\}} \cup \{i_b\} \cup \{y\} \cup F \cup \hat A_2$ and $A_1 = A_1^*\setminus{\{i_c\}} \cup \{i_c\} \cup \{x\} \cup \hat A_1$
\end{itemize}
So also in case in which there are not enough items with value $c$ for the frozen player to reach the constraint defined in \ref{eq:constraint-value-freeze-a-b-c}, we can obtain EFX allocation.

\chapter{Three Players}
\section{Constraint on the Values}
In the next pages I am going to show how to obtain an EFX allocation with three players and three values with the constraint that $c\ge a\mod b$. This constraint was not present in the discussion over two players, so why do I add it now? The answer is in how we deal with the fact that the frozen player envied the non frozen one in the two player case: it was enough to invert the problematic assignment. With the three-player case, this is no longer possible: let's consider the problematic assignment in which $p_1$ is frozen, $p_2$ and $p_3$ envies him and still can take other items valued $b$, in such a case both $p_2$ and $p_3$ by following the approach described in the case of two players could have to take items valued $c$ before finishing the ones valued $b$ if $c< a\mod b$, so we could have that not only $p_1$ starts to envy the non-frozen players, but also that the non-frozen player starts to envy each other and this is not solvable by changing the problematic assignment.
\section{Redefinition of Problematic Assignment}
For two players we have that only when the frozen player values remaining items as $b$, we could have a non EFX allocation by following the maximum matching assignments because in the end the frozen player could envy the non frozen player. In order to see what happens with three players we have to consider the different cases in which the frozen player values all the remaining items as $c$, let's consider that during the assignments that leads to freeze player $p_1$ he took item $x$, $p_2$ took item $y$ and $p_3$ took item $z$: 
\begin{itemize}
    \item $v_1(x) = a, v_2(x) = a, v_2(y) = b, v_3(x) = a, v_3(z) = c$: in this case $p_3$ needs exactly $\floor{a-c}{c}$ items that have value $c$ for $p_1$ and $p_3$, so $p_1$ when unfrozen will not envy $p_3$ (we have to notice that $v_1(z) = c$ otherwise we would have assigned $x$ to $p_3$ and $z$ to $p_1$ since $a + b + b > a + b + c$). Instead $p_2$ needs in the worst case (in the case in which he values all the remaining items as $c$) $\floor{a-b}{c}$ items, that have value lower or equal to $a-b-c$ for $p_1$, so also in this case $p_1$ will not envy $p_2$.
    
    \item $v_1(x) = a, v_2(x) = b, v_2(y) = c, v_3(x) = b, v_3(z) = c$: in this case both $p_2$ and $p_3$ have to take $\floor{b-c}{c}$ items. We have to see if $\floor{b-c}{c}\le \floor{a-w}{c}$ holds where $w=\max\{v_1(y), v_1(z)\}$. If $w=c$, than surely the condition above holds, instead in the case of $w=b$ let's say that $v_1(y) = b$, than we have that $v_1(x) + v_2(y) \ge v_1(y) + v_2(x)$ because of the maximum matching rule, so $a + c \ge 2b$. So from the precedent constraint we can write that $a -b \ge b-c$, so we have that $\floor{a-b}{c} \ge \floor{b-c}{c}$. So in both cases, we have that the number of items obtained by the frozen player is lower to the ones needed to $p_1$ to envy one of the other players when unfrozen.
    
    \item $v_1(x) = b, v_2(x) = b, v_2(y) = c, v_3(x) = b, v_3(z) = c$: in this case the frozen player will take $\floor{b-c}{c}$ items, so $p_1$ will envy no player when unfrozen because $v_1(y) = v_1(z) = c$ for the maximum matching assignment.
\end{itemize}
\section{We Cannot Use Maximum Matching}
In the case of three players we cannot use the maximum matching technique to assign the items while a player is frozen in one of the problematic assignments because we cannot control the value for the frozen player of both the players. In table \ref{table:example-maximum-matching-problem} there is an example that shows an allocation in bold obtained by choosing the items with the maximum matching technique while $p_1$ is frozen. As we can see also if we invert the assignment by freezing $p_2$ we can obtain the same assignment that leads the frozen player to envy $p_3$. In this particular case, by considering $a=400, b=100$ and $c=40$, in the end we have that $v_1(A_1) = a + c = 440$ while $v_1(A_3) - \min_{x \in A_3} v_1(x) = 5b + c -c = 500$, so we obtain a non EFX allocation. 
\begin{table}[h]
\centering
\begin{tabular}{lllllllllllllll}
                            & $i_1$                           & $i_2$                           & $i_3$                           & $i_4$                           & $i_5$                           & $i_6$                           & $i_7$                           & $i_8$                           & $i_9$                           & $i_{10}$                        & $i_{11}$                        & $i_{12}$                        & $i_{13}$                        & $i_{14}$                        \\ \hline
\multicolumn{1}{|l|}{$p_1$} & \multicolumn{1}{l|}{\textbf{a}} & \multicolumn{1}{l|}{b}          & \multicolumn{1}{l|}{b}          & \multicolumn{1}{l|}{b}          & \multicolumn{1}{l|}{c}          & \multicolumn{1}{l|}{b}          & \multicolumn{1}{l|}{c}          & \multicolumn{1}{l|}{b}          & \multicolumn{1}{l|}{c}          & \multicolumn{1}{l|}{b}          & \multicolumn{1}{l|}{c}          & \multicolumn{1}{l|}{c}          & \multicolumn{1}{l|}{c}          & \multicolumn{1}{l|}{\textbf{c}} \\ \hline
\multicolumn{1}{|l|}{$p_2$} & \multicolumn{1}{l|}{a}          & \multicolumn{1}{l|}{\textbf{b}} & \multicolumn{1}{l|}{b}          & \multicolumn{1}{l|}{b}          & \multicolumn{1}{l|}{\textbf{b}} & \multicolumn{1}{l|}{b}          & \multicolumn{1}{l|}{\textbf{b}} & \multicolumn{1}{l|}{c}          & \multicolumn{1}{l|}{\textbf{c}} & \multicolumn{1}{l|}{c}          & \multicolumn{1}{l|}{\textbf{c}} & \multicolumn{1}{l|}{c}          & \multicolumn{1}{l|}{\textbf{c}} & \multicolumn{1}{l|}{c}          \\ \hline
\multicolumn{1}{|l|}{$p_3$} & \multicolumn{1}{l|}{b}          & \multicolumn{1}{l|}{b}          & \multicolumn{1}{l|}{\textbf{b}} & \multicolumn{1}{l|}{\textbf{b}} & \multicolumn{1}{l|}{b}          & \multicolumn{1}{l|}{\textbf{b}} & \multicolumn{1}{l|}{b}          & \multicolumn{1}{l|}{\textbf{b}} & \multicolumn{1}{l|}{c}          & \multicolumn{1}{l|}{\textbf{b}} & \multicolumn{1}{l|}{c}          & \multicolumn{1}{l|}{\textbf{c}} & \multicolumn{1}{l|}{c}          & \multicolumn{1}{l|}{c}          \\ \hline
\end{tabular}
\caption{An assignment that shows the problem of using maximum matching while a player is frozen because of a problematic assignment}
\label{table:example-maximum-matching-problem}
\end{table}

\section{First Problematic Assignment}
\begin{table}[h]
\centering
\begin{tabular}{|l|l|l|l||l|l|l|l|}
\hline
      &            &            &            & x & y & z & w \\ \hline
$p_1$ & \textbf{a} & b          & c          & b & b & c & c \\ \hline
$p_2$ & a          & \textbf{b} & c          & b & c & b & c \\ \hline
$p_3$ & b          & c          & \textbf{c} & c & c & c & c \\ \hline
\end{tabular}
\caption{}
\label{table-3-players-aab}
\end{table}
In the table \ref{table-3-players-aab} the letters above each column are used in order to represent the type and also to represent the number of such items . Let's consider that we assign the items in bold, than we have to consider different cases:
\begin{itemize}
    \item $w \ge \left \lfloor \frac{b}{c}\right \rfloor - 1$: in this case we can assign $\left \lfloor \frac{b}{c}\right \rfloor - 1$ items of type $w$ to $p_3$ so that he does no longer envies $p_1$. Now we have the following situation for $p_1$:
    \begin{align*}
        v_1(A_2) &= v_1(A_2^*) + b\\
        v_1(A_3) &= v_1(A_3^*)  +\left \lfloor \frac{b}{c}\right \rfloor c \le v_1(A_3^*)  + b \\
    \end{align*}
    so for $p_1$ we can threat the other two players in the same manner. Now we assign to $p_2$ and $p_3$ items of type $x$, $z$ and $y$. Since $c \ge a\%b$ $p_2$ needs only items of value $b$ to reach the constraint so we can consider the following cases:
    \begin{itemize}
        \item $ \left \lfloor \frac{x+z}{2}\right \rfloor \ge  \left \lfloor \frac{a-b}{b}\right \rfloor$: in this case is enough to let take the same number of items of each type to both $p_2$ and $p_3$ till reaching $\left \lfloor \frac{a-b}{b}\right \rfloor$ items. It is easy to see that $p_1$ will not envy $p_2$ or $p_3$ since $a-b\ge v_2(F_2) \ge v_1(F_2)$.
        \item in the case in which the condition above does not hold, than we have that $p_2$ and $p_3$ need to take $\hat y$ items of type $y$ and this is not a problem if
        \begin{equation*}
            \begin{cases}
                a -b \ge  \left \lfloor \frac{x+z}{2}\right \rfloor b + \hat{y}c\\
                a -b \ge  \left \lfloor \frac{x}{2} + 1\right \rfloor b +  \left \lfloor \frac{z}{2} - 1\right \rfloor c + \hat{y}b
            \end{cases}
        \end{equation*}
        So we have a non EFX allocation if $\hat{y} > z$, in this case is enough to invert the assignment to the one shown in table \ref{table-3-players-aab-invert-w-greater-second-case} since is like considering $p_1$ inverted with $p_2$, so exchanging $y$ with $z$.
        \begin{table}[h]
            \centering
                \begin{tabular}{|l|l|l|l|}
                \hline
                $p_1$ & a          & \textbf{b} & c           \\ \hline
                $p_2$ & \textbf{a} & b          & c           \\ \hline
                $p_3$ & b          & c          & \textbf{c}  \\ \hline
                \end{tabular}
            \caption{}
            \label{table-3-players-aab-invert-w-greater-second-case}
        \end{table}
    \end{itemize}
    
    \item $w < \left \lfloor \frac{b}{c}\right \rfloor - 1 $: in this case we will not assign all the items of type $w$ to $p_3$. So in this case we have that $p_1$ envies no one, $p_2$ envies only $p_1$ and the same $p_3$. In order to remove the fact that $p_2$ and $p_3$ still envy $p_1$ we need to assign to $p_2$ $\left \lfloor \frac{a-b}{b} \right \rfloor $ items of type $x$ or $z$ and other $\left \lfloor \frac{a-b - \floor{x+z}{2}b}{c} \right \rfloor$ items of type $y$ or $w$ if the items of type $x$ and $z$ are not enough; instead $p_3$ needs to obtain $\left\lfloor \frac{b}{c}\right \rfloor -1$ items of any type. So let's consider the following cases:
    \begin{itemize}
        \item $\left \lfloor \frac{x+z}{2} \right \rfloor\ge \left \lfloor \frac{a-b}{b} \right \rfloor$ and $\left \lfloor \frac{x+z}{2} \right \rfloor\ge \left\lfloor \frac{b}{c}\right \rfloor -1 $: in this case by assigning to $p_2$ and $p_3$ the items of type $x$ and $z$ till $p_2$ does not envy $p_1$ leads to an EFX allocation cause for $p_1$ the value obtained by the two players is lower or equal to $a-b$ as for $p_2$, for $p_2$ the obtained values is larger or equal to $a-b-c$ and for $p_3$ we have obtained the required number of items. 
         \item $\left \lfloor \frac{x+z}{2} \right \rfloor\ge \left \lfloor \frac{a-b}{b} \right \rfloor$ and $\left \lfloor \frac{x+z}{2} \right \rfloor< \left\lfloor \frac{b}{c}\right \rfloor -1 $: in this case we do not always obtain an EFX allocation because in order to have $p_3$ to not envy $p_1$ we could have to give to $p_3$ too many items that have value $b$ for $p_1$, so $p_1$ would envy him when unfrozen. 
         %In this case we can invert the initial assignment to the one shown in table 
         We can consider the following two case:
         \begin{itemize}
            \item $y\ge \floor{z'}{2}$: in this case we can change the assignment to the one show in table \ref{table-3-players-aab-invert-w-lower-second-case-p3p1p2}. Than we can assign the items as follows: assign half of the $x$ items to $p_1$ and half to $p_2$ and than assign $\frac{z'}{2}$ items of type $z$ to $p_2$ and $\frac{z'}{2}$ items of type $y$ to $p_1$. If $x$ is odd, than also $z'$ is, in this case we have to assign the $x$ odd item to $p_1$ and the $z'$ odd item to $p_2$ in order to give each player $\floor{a-b}{b}$ items that he values $b$. So we have assigned $\floor{x+z'}{2}$ items to each player and now both no longer envy $p_3$, moreover $p_3$ will not envy them since $\floor{x+z'}{2} < \floor{b}{c} - 1$.
            \item $y < \floor{z'}{2}$: in this case we can avoid to change the assignment since, if $y< \frac{z'}{2}$ than $y < \left \lfloor \frac{a-b-\frac{x}{2}}{b}\right \rfloor$.
             %, so also if $p_2$ and $p_3$ take all the items of type $y$, than the value obtained by the two players for $p_1$ will still be lower than $a-b$, so $p_1$ will not envy any of the other two players. 
            In this case we have to do the following assignment: assign to $p_2$ $\frac{x}{2}$ items of type $x$, $y$ items of type $z$ and than $\frac{z'}{2}-y + \floor{b-c-\frac{x}{2}c - \frac{z'}{2}c}{c}$ items of type $z$, instead we have to assign to $p_3$: $\frac{x}{2}$ items of type $x$, $y$ items of type $y$ and than $\frac{z'}{2}-y + \floor{b-c-\frac{x}{2}c - \frac{z'}{2}c}{c}$ items of type $z$. As we can see $p_1$ will not envy $p_2$ since 
            \begin{align*}
                v_1(F_2) &= \frac{x}{2}b + \frac{z'}{2}c + \floor{b-c-\frac{x}{2}c - \frac{z'}{2}c}{c} c \\
                & \le a-2b + b - c = a- b - c
            \end{align*}    
            since we are considering that $y<\floor{z'}{2}$, so $\floor{z'}{2} \ge 1$ we have $ \frac{x}{2}b \le a-2b$ and we also have that $\frac{z'}{2}c + \floor{b-c-\frac{x}{2}c - \frac{z'}{2}c}{c} c \le b-c$ . 
            $p_1$ will also not envy $p_3$ as we can see by the next equation:
            \begin{align*}
                v_1(F_3) &= \frac{x}{2}b + yb + (\frac{z'}{2}-y)c \floor{b-c-\frac{x}{2}c - \frac{z'}{2}c}{c} c \\
                & \le a-2b + b - c = a- b - c
            \end{align*}
            since, as before,  $y<\frac{z'}{2}$, we have $\frac{x}{2}b + yb\le a-2b$ and we also have that $(\frac{z'}{2}-y)c \floor{b-c-\frac{x}{2}c - \frac{z'}{2}c}{c} c \le b-c$.
            We can notice that if $x$ is odd, than we still have the same conditions if we consider that we give the $x$ odd item to $p_3$ and that $y < \floor{a-b}{b} - \floor{x}{2} - 1$, if this does not hold, than we can see that we obtain an EFX allocation by following the approach described in the precedent case.
         \end{itemize}
         
         \begin{table}[h]
            \centering
                \begin{tabular}{|l|l|l|l|}
                \hline
                $p_1$ & a               & \textbf{b}    & c             \\ \hline
                $p_2$ & a               & b             & \textbf{b}    \\ \hline
                $p_3$ & \textbf{b}      & c             & c             \\ \hline
                \end{tabular}
            \caption{}
            \label{table-3-players-aab-invert-w-lower-second-case-p3p1p2}
        \end{table}
        \item $\floor{x + z}{2} < \floor{a-b}{b}$: in this case we must take also items of type $y$. Let's consider that we give $y'$ items to the non frozen players of type $y$, than we must have that two or none between $x$, $y'$ and $z$ are odd because $x+y'+z$ has to be even in order to split the items between the two non frozen players. Let's deal with the possible case one per time:
        
        \begin{itemize}
            \item $\floor{y'}{2} \le \floor{z}{2}$: in order to solve this problem we have to do the following assignments: to $p_2$ and $p_3$ $\floor{x}{2}$ items of type $x$, than we assign one item of the two types that have an odd number of items by giving the one (there is always at least one) with value $b$ for $p_2$ to $p_2$, than we assign to each player $\floor{z}{2}$ items of type $z$ and in the end we assign $\floor{y'}{2}$ items of type $y$ to each player. In all cases, since the items of type $z$ given to each player are more than the ones of type $y$, the value for $p_1$ of the value obtained by the two players will be lower or equal to the value obtained by $p_2$ that is lower or equal to $a-b$.
            \item $\floor{y'}{2} > \floor{z}{2}$: we can see that in this case we can do the same thing done before by exchanging $p_1$, $p_2$ and $z$ and $y'$.
        \end{itemize}
        In all the above cases I have assumed that the number of items obtained by $p_3$ are greater or equal to $\floor{b}{c} -1$ because when we assign to $p_2$ items with value $c$ for him (items of type $y$ and $w$), we are assigning them in order give him at least a value of $b-c$ (because we have the constraint $c\ge a \mod b $), so also considering only these items, $p_3$ took enough items to no longer envy $p_1$.
    \end{itemize}
 
 \end{itemize}
  We can notice that for how we give the items to the two non frozen players, we will never have that one player envies the other, so we will never have to freeze a player in this phase. Instead after that the frozen player $p_i$ is unfrozen, we could have another assignment that can lead to a freeze: in this case we can see that $p_3$ is not a cause of this freeze since all the remaining items have value $c$, so we will freeze one player among $p_1$ and $p_2$. If this happens, than we have to freeze the player that has not yet been frozen. Let's consider that in the first freeze we froze $p_1$, than when $p_1$ is unfrozen he is envy-free to the other two players, while for $p_2$ we have that $v_2(A_2) \ge v_2(A_1) - c$, so by freezing $p_2$ we ensure the fact that $p_1$ takes value lower than $p_2$ with a difference of at most a $c$, while $p_2$ takes items with larger or equal value to the ones obtained by $p_1$, so we are ensuring the fact that $p_1$ and $p_2$ are EFX.

\section{Second Problematic Assignment}
\begin{table}[h]
\centering
\begin{tabular}{|l|l|l|l||l|l|}
\hline
      & $i_1$           & $i_2$      & $i_3$        & z & w \\ \hline
$p_1$ & \textbf{a}      & b          & b            & b & c \\ \hline
$p_2$ & b               & \textbf{c} & c            & c & c \\ \hline
$p_3$ & b               & c          & \textbf{c}   & c & c \\ \hline
\end{tabular}
\caption{}
\label{table-3-players-abb-bcc-bcc}
\end{table}
In this case is easy to see that $p_2$ and $p_3$ both need $\floor{b-c}{c}$ items of any remaining type. We can differentiate two cases by considering the number $\hat z$ of items of type $z$ that we have to assign to $p_2$ and $p_3$ to reach the required number of items
\begin{itemize}
    \item $\floor{w}{2}c + \hat z b \le a-b$: where $\hat z = \floor{b-c-\floor{w}{2}c}{c}$, in this case is enough to assign to $p_2$ and $p_3$ first the items of type $w$, than the items of type $z$ in equal values.
    \item $\floor{w}{2}c + \hat z b > a-b$: where $\hat z = \floor{b-c-\floor{w}{2}c}{c}$, in this case we have to invert the assignment so that we freeze player $p_2$ rather than $p_1$. Now we have that $p_1$ and $p_3$ will envy $p_2$ for different values. By assigning the items of type $z$ to $p_1$ and of type $w$ to $p_3$ till $p_3$ reaches $\floor{b-c}{c}$ items, we have an EFX allocation since 
    \begin{itemize}
        \item $p_1$ will not envy $p_2$ since he surely took more than $a-b$ value and equal or higher value than $p_3$ while $p_2$ was frozen,
        \item $p_2$ will not envy the other players since they took at most $b-c$ value while $p_2$ was frozen,
        \item $p_3$ will not envy $p_1$ since they take the same items and will not envy $p_2$ by considering that $p_3$ ha precedence over the last iteration assignment.
    \end{itemize}
\end{itemize}


\section{Third Problematic Assignment}

\begin{table}[h]
\centering
\begin{tabular}{|l|l|l|l||l|l|l|l|l|l|l|l|}
\hline
      &                 &               &               & x & y & z & j & k & l & w & v \\ \hline
$p_1$ & \textbf{a}     & b             & b             & b & c & b & b & c & b & c & c \\ \hline
$p_2$ & a               & \textbf{b}    & b             & b & b & c & b & b & c & c & c \\ \hline
$p_3$ & a               & b             & \textbf{b}    & c & c & c & b & b & b & c & b \\ \hline
\end{tabular}
\end{table}
Let's assume that 
$$x + y + k + j \ge j + k + l + v \ge x + z + j + l$$
so that the second player has the higher number of items that he values $b$, followed by $p_3$ and $p_1$. Now we can show that $p_1$ will never envy the other two players when he is unfrozen if we assign the items as follows till both player have reached $v_i(F_i) \ge a-b-c$: we assign first the items of type $y$ and $x$ to $p_2$, and the items of type $v$ and $l$ to $p_3$. Since we have assumed that $x + y + k + j \ge j + k + l + v $, we have $x + y  \ge  l + v $ so $p_3$ will finish first the items of type $l$ and $v$, so while $p_2$ is still taking items of type $x$ and $y$ he will star taking items of type $k$ and $l$. Now we can have two different cases:
\begin{itemize}
    \item $x + y \ge l + v + j + k $: in this case $p_3$ will finish all the items that he values $b$ ($l,v,j$ and $k$) while $p_2$ is still taking items of type $x$ and $y$. So after that $p_3$ has finished these items we will start assigning the same type of items to the two non frozen players in the following order $x$,$y$, $w$ and $z$. In this case is easy to see that $p_1$ will not envy the non frozen players when he will be unfrozen because $p_3$ takes all the items that he values $b$ and by the initial assumption we have that $j + k + l + v \ge x + z + j + l$, so $p_3$ will surely achieve $a-b-c$ before $p_1$ does it considering the bundle obtained by one of the two non frozen players.
    \item $x + y < l + v + j + k $: in this case $p_2$ will finish all the items of type $x$ and $y$ when there are still left other items of type $j$ and $k$. We define to be $k'$ and $j'$ the items of type $k$ and $j$ respectively taken by $p_3$ while $p_2$ is still taking items of type $x$ and $y$, and we define $rem_k$ and $rem_j$ to be $k-k'$ and $j-j'$ respectively. We will divide $rem_k$ and $rem_j$ equally among the two non frozen players and if the remaining $j$ and $k$ items are odd, we give that item to $p_2$ and we freeze him till $p_3$ obtains $\floor{b-c}{c}$ items among the remaining ones that are of type $w$ and $z$. 
    Both in case of a second freeze and in case of no second freeze while $p_1$ is frozen, we have that at most $p_1$ will envy one of the two other players and not both. In table \ref{table:abb-abb-abb-assignment-in-the-case-of-second-freeze} and  \ref{table:abb-abb-abb-assignment-in-the-case-of-no-second-freeze} we can see the assignments done for the case in which we have a freeze and the case in which we have no freeze, where we consider $rem_z$ to be the number of remaining items of type $z$ after that $p_2$ has been unfrozen in the case of second freeze.
    We can see that for $p_1$ to envy both other players we must have that both $v_1(F_2) > v_2(F_2)$ and $v_1(F_3) > v_3(F_3)$ are true. In the following equations we can see that this is impossible in the case of second freeze while $p_1$ is frozen, but this can also consider the case in which we have no second freeze because is enough to have $rem_z = z$.
    
    \begin{align*}
        v_1(F_2) &= xb + yc + \floor{rem_j}{2}b + \floor{rem_k}{2}c + b + \floor{rem_z}{2}b + \floor{rem_w}{2}c\\
        v_2(F_2) &= xb + yb + \floor{rem_j}{2}b + \floor{rem_k}{2}b + b + \floor{rem_z}{2}c + \floor{rem_w}{2}c\\
        \\
        v_1(F_3) &= vc + lb + j'b + k'c + \floor{rem_j}{2}b + \floor{rem_k}{2}c + c + (z-rem_z)b \floor{rem_z}{2}b + \floor{rem_w}{2}c\\
        v_3(F_3) &= vb + lb + j'b + k'b + \floor{rem_j}{2}b + \floor{rem_k}{2}b + c + (z-rem_z)c \floor{rem_z}{2}c + \floor{rem_w}{2}c
    \end{align*}
       
    \begin{align*}
        &v_1(F_2) > v_2(F_2)\implies \floor{rem_z}{2} > y + \floor{rem_k}{2} \\
        &v_1(F_3) > v_3(F_3)\implies \floor{rem_z}{2} + z-rem_z > v + k - rem_k+  \floor{rem_k}{2} 
    \end{align*}
    By summing the two inequalities we obtain
    \begin{align*}
        &2\floor{rem_z}{2} + z-rem_z > y + v + k - rem_k+  2\floor{rem_k}{2} 
    \end{align*}
    that is impossible because of the initial assumption, indeed $z \le k + v + y$.
    So $p_1$ will at most envy one of the other two players. This will happen only when the two non frozen player are not able to reach $a-b-c$ with the only items of type $b$ because otherwise is immediate that the value for $p_1$ of the bundles obtained by the other two players while he was frozen is lower or equal to the value obtained by the two non frozen players that is lower or equal to $a-b-c$. So in the case in which $p_1$ envies one of the two other players we can swap the bundles obtained from the problematic iteration till the iteration in which we unfreeze $p_1$ of these two players.
    %Let's start considering the case in which we have a second freeze, than we have assigned to $p_2$ the following items $x$, $y$, $\floor{rem_k}{2}$ of type $k$, $\floor{rem_j}{2}$ of type $j$, the odd item (type $k$ or $j$ that provokes the second freeze), $\floor{z'}{2}$ of type $z$ and $\floor{w}{2}$ of type $w$; instead $p_3$ takes $v$, $l$, $x + y - v- l$ of type $j$ and $k$,  $\floor{rem_k}{2}$ of type $k$, $\floor{rem_j}{2}$ of type $j$ an item of type $z$ or $w$, $z-z'$  items of type $z$ and the items of type $w$ while $p_2$ is frozen 
    
    \begin{table}[h]
        \begin{tabular}{|l|l|l|l|l|l|l|l|l|l|l|l|}
            \hline
            $p_2$ & $x$ & $y$ &      &    &   $\floor{rem_k}{2}$ & $\floor{rem_j}{2}$ & odd k or j & frozen & $\floor{rem_w}{2}$ & $\floor{rem_z}{2}$ & z odd \\ \hline
            $p_3$ & $v$ & $l$ & $k'$ & $j'$ & $\floor{rem_k}{2}$ & $\floor{rem_j}{2}$ & w or z     & w, z   & $\floor{rem_w}{2}$ & $\floor{rem_z}{2}$ & w odd \\ \hline
        \end{tabular}
        \caption{Assignment in the case of second freeze while $p_1$ is frozen}
        \label{table:abb-abb-abb-assignment-in-the-case-of-second-freeze}
    \end{table}
    
    \begin{table}[h]
        \begin{tabular}{|l|l|l|l|l|l|l|l|l|l|}
            \hline
            $p_2$ & $x$ & $y$ &      &    &   $\floor{rem_k}{2}$ & $\floor{rem_j}{2}$ & $\floor{w}{2}$ & $\floor{z}{2}$ & z odd \\ \hline
            $p_3$ & $v$ & $l$ & $k'$ & $j'$ & $\floor{rem_k}{2}$ & $\floor{rem_j}{2}$ & $\floor{w}{2}$ & $\floor{z}{2}$ & w odd \\ \hline
        \end{tabular}
        \caption{Assignment in the case of no second freeze while $p_1$ is frozen}
        \label{table:abb-abb-abb-assignment-in-the-case-of-no-second-freeze}
    \end{table}
    
\end{itemize}

Now let's consider first the case in which we had no swap after the unfreeze of $p_1$ and than the case in which we have it:

\begin{itemize}
    \item in the cases in which we had no swap after the unfreeze of $p_1$ we have the following situation 
    \begin{align*}
    &v_1(\hat A_1) + a \ge v_1(\hat A_2) + b + v_1(F_2)\\
    &v_1(\hat A_1) + a \ge v_1(\hat A_3) + b + v_1(F_3)\\
    \\
    &v_2(\hat A_2) + b + v_2(F_2) \ge v_2(\hat A_1) + a - c \\
    &v_2(\hat A_2) + b + v_2(F_2) \ge v_2(\hat A_3) + b + v_2(F_3)\\
    \\
    &v_3(\hat A_3) + b + v_3(F_3) \ge v_3(\hat A_1) + a - c\\
    &v_3(\hat A_3) + b + v_3(F_3) \ge v_3(\hat A_2) + b + v_3(F_2) - c\\
    \end{align*}
Where in the last inequality we have the last $-c$ only if we had a second freeze.

After this we have that in the last iteration we will assign the items with priority to $p_2$ and $p_3$, and among these two players with priority to the player who values more the item if we had no second freeze, or to $p_3$ if we had a second freeze for which we froze $p_2$. The precedent assertion works only when the last item has value $c$ for $p_1$, since he is EF towards the other two players till the last iteration, or $p_2$ and $p_3$ did not took any item of type $z$ while frozen (so in the case in which the last item is of type $j, k, x, y, l$ and $v$) since in this case each item obtained by the frozen players while $p_1$ was frozen, has value higher or equal, for the player who took it, to the value that has for $p_1$, so in the case in which the last item values $b$ for $p_1$ we can have two cases: $\min_{x\in A_i} v_1(x) = b$ or $\min_{x\in A_i} v_1(x) = c$, the first case is easy for the definition of EFX, the second case instead has more to argue. In the case in which  $\min_{x\in A_i} v_1(x) = c$, we have that or this item has been obtained while $p_1$ was frozen, and this would imply that $v_1(F_i) \le a - b - b + c = a-2b+c$ so we can assign the last item to player $i$ keeping $p_1$ EFX towards $p_i$, or is associated to an item that $p_1$ took with value $b$, so we can still assign the last item to $p_i$ by keeping $p_1$ EFX towards $p_i$. Instead if the last item is of type $z$, than we could have assigned some items of type $z$ to $p_i$, in this case we can't assert the same things that I have told before for the case in which $\min_{x\in A_i} v_1(x) = c$.  Let's divide the cases based on the values of each player for it's set of items obtained from the iteration in which we freeze $p_1$ and let's call this set for a generic player $p_i$ as $\bar{A_i}$: 
\begin{itemize}
    \item $v_2(\bar A_2) \ge v_2(\bar A_1)$ and $v_3(\bar A_3) \ge v_3(\bar A_1)$: in this case we have to assign the last item to $p_1$ and if the last items are two, if there was a second freeze, to the non frozen player, otherwise to one between $p_2$ and $p_3$. We can notice that since the item is valued $c$ by $p_2$ and $p_3$ we obtain an EFX allocation.
    \item $v_2(\bar A_2) \ge v_2(\bar A_1)$ and $v_3(\bar A_3) \ge v_3(\bar A_1) - c$: in this case we can notice that since $p_2$ took more value than the value obtained by $p_1$ or he had to wait $p_3$ ($x+y\ge j+k+v+l$) or after the freeze he took some item with value $b$ while in the same iteration $p_1$ took an item valued $c$ by $p_2$; in both cases we have that there has not been a second freeze between $p_2$ and $p_3$, so we have that $p_2$ and $p_3$ are EF till now. If there are two last items, we assign them to $p_3$ and $p_1$ since $p_2$ is EF towards both of them, instead if is only one item and we have that $v_1(\bar A_3) + b - c \le v_1(\bar A_1)$, we can assign the last item to $p_3$. The last case to consider is when we have only one last item and $v_1(\bar A_3) + b - c > v_1(\bar A_1)$, in this case we can swap the bundles of $p_1$ and $p_3$ and assign the last item to $p_1$. This because we will obtain $N_1 = \bar A_3, N_2 = \bar A_2$ and $N_3 = \bar A_1$ such that $v_1(N_1) + b - c\ge v_1(N_3)$ and $v_3(N_3) \ge v_3(N_1)$.
    \item $v_2(\bar A_2) \ge v_2(\bar A_1) - c$ and $v_3(\bar A_3) \ge v_3(\bar A_1)$: in this case hold the same things as before.
    \item $v_2(\bar A_2) \ge v_2(\bar A_1) - c$ and $v_3(\bar A_3) \ge v_3(\bar A_1) - c$: this case can be handled like the precedent one, but in this case we have that $p_2$ and $p_3$ could not be EF. If there is only one last item and we have that for at least one between $p_2$ and $p_3$ we have that $v_1(\bar A_i) + b - c \le v_1(\bar A_1)$ than we assign the last item to $p_i$. If instead this does not hold, than we can exchange the bundles of $p_1$ and $p_i$ where $i = \argmax_{i\in \{2,3\}} v_1(\bar A_i)$. Than we can assign the last items to $p_1$ and $p_j$ (not $p_i$). Assigning the last item also to $p_j$ could provoke to not having $p_i$ EFX towards $p_j$ because of a second freeze in which we froze $p_j$, so in this case we change that assignment by freezing $p_i$ and not $p_j$. We can notice that the value $v_1(\bar A_i)$ will only increase with this last swap, because $p_i$ will take more items than before, so the exchange of bundles between $p_1$ and $p_i$ done before will not change. 
\end{itemize}
    \item Instead in the case in which we had to swap after the unfreeze of $p_1$ we have that
\begin{align*}
    &v_1(\hat A_1) + b + v_1(F_2) \ge v_1(\hat A_2) + a\\
    &v_1(\hat A_1) + b + v_1(F_2) \ge v_1(\hat A_3) + b + v_1(F_3)\\
    \\
    &v_2(\hat A_2) + a \ge v_2(\hat A_1) + b + v_2(F_2) \\
    &v_2(\hat A_2) + a \ge v_2(\hat A_3) + b + v_2(F_3)\\
    \\
    &v_3(\hat A_3) + b + v_3(F_3) \ge v_3(\hat A_1) + b + v_3(F_2) - c\\
    &v_3(\hat A_3) + b + v_3(F_3) \ge v_3(\hat A_2) + a - c\\
\end{align*}

where the $-c$ in the last two rows is the first one because of in the case of second freeze while $p_1$ was frozen, we have freeze $p_2$ but now that bundle is of $p_1$ and the second one because $p_2$ now has the item valued $a$ and $p_3$ has that $v_3(F_3) \ge a - b - c$. In this case we have that the precedence to take the last items is of $p_3$ because is the only non EF player towards the others. 

In this case we can notice that $p_1$ and $p_2$ are EF towards the other two players and also that this case only happens when the two non frozen player start to take items of type $z$, so we can assert that the only two types of items that will be as last item are $w$ and $z$. In the case of last item of type $w$ is enough to assign it with precedence to $p_3$, instead in case of last item of type $z$, if there are two of such items, we can assign one to $p_1$ and one to $p_3$. Instead in the case of only one last item of type $z$ we have to deal with the fact that $p_3$ needs to take the last item and also that that item values $b$ for $p_1$. In the case in which in $F_2$ there is an item valued $c$ by $p_1$, we exchange that item with the last item and than we give it to $p_3$. Now I'll show that the bundle obtained while $p_1$ was frozen by the player who exchanged it with him has at least one item that $p_1$ values $c$: let's start considering $F_2$, than if by absurd we consider that $rem_k = 0$ than we must have that $x + y \ge v + l + k$, but than also $y = 0$, so we would have that $x \ge v + l + k$ and this violates the first assumption that for which $k + y \ge z + l$ and so $k \ge z + l$ and also $k + v \ge x + z$, but before we wrote that $x \ge v + l + k$. Instead in the case of $F_3$ we have that to have no item valued $c$, we must have that $k = 0, v = 0$ and this violates the initial assumption for which $k + v \ge x + z$.
\end{itemize}



\section{Problematic Assignments Where There Is a Player That Does Not Envy the Other}
\begin{table}[h]
\centering
\begin{tabular}{|l|l|l|l|l|l|l|l|l|l|l|l|l|}
\hline
      & d & e & f & g & j & k & y & x & v & l & z & w \\ \hline
$p_1$ & b & b & c & c & b & c & c & b & c & b & b & c \\ \hline
$p_2$ & b & c & b & c & b & b & b & b & c & c & c & c \\ \hline
$p_3$ & a & a & a & a & b & b & c & c & b & b & c & c \\ \hline
\end{tabular}
\caption{Types of items after a problematic assignment}
\end{table}
In this section I am going to describe the approach used to obtain EFX allocation when we have a problematic assignment in which a player ($p_3$ without loss of generality) does not envy the item obtained by the other two players. In the beginning of the proof I will show the case in which each type of item has an even number of elements, than I will show that in the case in which this not hold, we can obtain the same characteristics with other assignments. 
\paragraph{Even number of items} Let's consider the case in which we have even number of items for each remaining type after the problematic assignment, in this case we can assign the items to $p_3$ by letting him choose the one with higher value for him and than assign the same type of item to the other non frozen player ($p_2$ without loss of generality). By doing so we have that $v_1(F_2) = v_1(F_3)$, so if when unfrozen $p_1$ envies $p_3$ than he envies also $p_2$. We have two cases now:
\begin{itemize}
\item when unfrozen $v_1(F_2) \le a-b$: in this case we have that all players are EF towards each other except for $p_2$ that has $v_2(\bar A_1) \ge v_2(\bar A_2)> v_2(\bar A_1) - c$, where $\bar A_i$ is the set of items obtained by player $i$ from the problematic assignment on.
\item when unfrozen $v_1(F_2) > a-b$: in this case we have to invert the assignment and give $\bar A_1$ to $p_2$ and $\bar A_2$ to $p_1$ and after this we have that all the players are EF towards each other.
\end{itemize}
Now we have to deal with the fact that the player who has $F_2$ ($p_2$ in the first case and $p_1$ in the second case) did not took the items by first taking all the items with value $b$ and than the ones with value $c$ because he took always the same item took by $p_3$, so we could have some problems when assigning the last items. As first thing we can notice that we can avoid to give the last item to $p_3$ because he is EF towards $p_2$, $p_1$  and took the items in the right order, so we have to deal only on the case in which we have a single last item that we have to give to $p_1$ or $p_2$. The main problem raises when the last item is valued by both $p_1$ and $p_2$ as $b$. Now in order to show how to deal with this case I am going to consider the case in which there was no swap after that $p_1$ has been unfrozen, in the other case is enough to exchange $p_1$ with $p_2$. So if we have that $\min_{x\in A_2} v_1(x) = b$ than we can assign the item to $p_2$ and obtain an EFX allocation directly, instead in the case in which $\min_{x\in A_2} v_1(x) = c$ than we have to consider when the item $g$ valued $c$ by $p_1$ has been obtained: if it has been obtained before or after the problematic assignment, than surely $p_1$ took in the same iteration an item valued $b$ or higher, so by assigning the last item to $p_2$ we still have an EFX allocation. In the case in which the item valued $c$ by $p_1$ has been obtained when $p_1$ was frozen, than we have that the last item can be of two types: 
\begin{itemize}
    \item $g = (c,b,\star)$: in this case or we have that $\exists f=(b,c,\star)\in F_2$ or $v_1(F_2) \le a-2b + c$. In the latter case we can assign the last item directly to $p_2$ by keeping an EFX allocation, in the first case instead we can assign to $p_1$ $f$, while to $p_2$ we assign $g$. In this manner we decrease the value for $p_1$ and $p_2$ of the other player set of items of $b-c$, so we can assign to $p_2$ the last item. We can also notice that the value of $A_2$ will not increase for $p_3$ since the last item has surely lower or equal value to the item removed. Instead for what concerns $p_1$ for $p_3$, since $p_3$ did not envied him when $p_1$ was frozen, $p_3$ has $v_3(A_2)\le v_3(A_3\setminus F_3)$ and $v_3(F_3)\ge v_3(f)$, so also for $p_3$ this is still an EFX allocation.
    \item $g = (c,c,\star)$: this last case is the case in which all the items in $A_2$ valued $c$ by $p_1$ value also for him $c$ and have been obtained while $p_1$ was frozen. In this case we give all (or at most $\floor{b}{c}$) such type of items to $p_1$ from $F_2$ and assign the last item to $p_2$. If there are not enough such items we are still assigning them to $p_1$ and than we obtain that $\min_{x \in A_2} v_1(x) = b $, so we can assign the last item to $p_2$. If instead there are such items, we obtain an EFX allocation because the value for $p_1$ of $A_2$ increases at most of $c$. Also in this case for $p_3$ this is not a problem because the items assigned to $p_1$ from $F_2$ have still value lower or equal to $v_3(F_3)$.
\end{itemize}
\paragraph{non even number of items} in the precedent paragraph I have shown that by giving the same items to $p_3$ and $p_2$ chosen by $p_3$ we obtain an EFX allocation in the end, the main characteristic of such assignments that lead to an EFX allocation is that $v_1(F_2) \ge v_1(F_3)$ and that $p_2$ and $p_3$ are EF. In this paragraph I am going to show ho to keep such constraints when we have no even number of items and, in the case in which is not possible, how to still obtain an EFX allocation. In the next part I am going to show how to deal with the most difficult cases, for all the ones that have not been discussed here, it means that or we can assign the odd $g$ item to $p_3$ and we have another one $f$ such that $v_2(f)\ge v_2(g)$ and $v_1(f) \le v_1(g)$ or the opposite.
\begin{itemize}
    \item $d$ is odd: \begin{itemize}
        \item $e,f,g,j,x=0$: in this case we can assign to $p_2$ one item between $k$ or $z$ and one between $k$ or $y$, while to $p_3$ we assign the odd $d$ item. In this case I have to highlight that $p_3$ will not envy $p_2$ because is impossible to have $2b\ge a$, indeed in this case we would have that no item is taken while $p_1$ is frozen because $b\ge a-b$.
        \item $e,f,g,j,x, l, z=0$: in this case we have to make a manual assignments and not follow the algorithm described in this subsection. In this case we have that the only types of items that are left are $d,k,y,w$ and $v$. In this case we can split the $d$ items between $p_2$ and $p_3$ and than assign the odd $d$ item to $p_3$, while one between $k$ or $y$ to $p_2$ (or in the case in which $k=y=0$ we assign $\floor{b}{c}$ items between $w$ and $v$ and in the case in which these item have value higher of $a$ for $p_3$, than invert such last assignment)
        \item $e,f,g,j,x,k,y=0$: in this case, since $f+k=0$ we have that also $l+e=0$ and so $l=e=0$. So in this case we only have the following items: $d,z,w$ and $v$ moreover, since this case has to differ from the precedent one, we have that $z\ne 0$ (before we have also considered the case in which $l=z=k=y=0$) so in this case we can freeze $p_2$ and not $p_1$ and assign to $p_1$ and $p_3$ each half of the items of type $d$; than we assign to $p_3$ the odd $d$ item and to $p_1$ an item of type $z$, than we continue to assign items of type $z$ to $p_2$ and of type $v$ to $p_3$. When we unfreeze $p_2$, he will not envy the other two players.
    \end{itemize}
    \item $e$ is odd:
    \begin{itemize}
        \item $z,x,l,g,j=0$ in this case we can assign to $p_2$ the odd $e$ item and an item of type $y$, while to $p_3$ we assign an item of type $f$ and one of type $v$ or $w$
        \item also $y=0$: in this case we assign to $p_2$ the odd $e$ item and an item of type $k$, while to $p_3$ we have to assign an item of type $f$ and one of type $v$ (not $w$).
        \item also $v = 0$: in this case we cannot follow the algorithm, but in this case we have very few types of items, so we can assign the items manually as follows: we start assigning to $p_2$ the items of type $k$ (if there are, $k$ in this case could be also $0$) and $f$, while to $p_3$ we assign the items of type $e$ and since we are considering that $f+k\ge l+e$ and that $l=0$, than $k+f\ge e$ and so the $v_1(F_2)\le v_1(F_3) \le v_2(F_2)\le a-b$. So when unfrozen $p_1$ will not envy $p_2$ or $p_3$. From this we can easily obtain an EFX allocation as described in the beginning of the subsection. 
        \item $z,x,l,g,j,f=0$: in this case we have to assign the items manually in the following manner: first we assign the items of type $d$ to both $p_2$ and $p_3$, than, since $f+k\ge l+e$ and so $k\ge e$, we assign the items of type $k$ to $p_2$ and the items of type $e$ to $p_3$, than we will assign the remaining items of type $y$ and $v$ respectively to $p_2$ and $p_3$. As before, we have that $v_1(F_2)\le v_1(F_3) \le v_2(F_2)\le a-b$ and so we can obtain an EFX allocation in the end.
    \end{itemize}
    \item $l,v,k=0$ and $j$ is odd: in this case we can assign the odd $j$ item to $p_3$ and $\floor{b}{c}$ items between the ones of type $x,z,w,y$ to $p_2$. In this case we can notice that $p_1$ will envy more $p_3$ but we still have that $v_1(F_2) > v_1(F_3) - c$, so by swapping with $p_2$ and not $p_3$ we still obtain EFX allocation since in the end $p_3$ does not take items in the last iteration.
    \item $x,z,v=0$ and $l$ is odd: in this case we assign the odd $l$ item to $p_3$ and $\floor{b}{c}$ items between the ones of type $w,y$ to $p_2$. In this case holds the same thing said in the precedent point.
    \item $y,x=0$ and $k$ odd: in this case we assign the odd $k$ item to $p_3$ and $\floor{b}{c}$ items between the ones of type $z,w$ to $p_2$. In this case holds the same thing said in the precedent point.
\end{itemize}
Till now I have not considered the fact that, when $p_2$ lost an item valued $a$, since $p_2$ is not taking the items following their value for him, we have that cause of the presence of some items valued $c$ b $p_2$ in $F_2$, we could reach a point in which by adding an item $g$ we have that $v_2(F_2) \le a-b-c$ and $v_2(F_2) +v_2(g)\ge a-b$ and $v_1(F_2) \le a-b-c$ and $v_1(F_2) +v_1(g)\ge a-b$. If such thing happen, than we cannot invert the assignment and obtain an EF allocation because the bundle given to $p_1$ would have value higher than $a$ for $p_2$. Such thing can happen after that $p_2$ took value $b$ for himself and $p_1$. In order to solve such problem we can divide in the following cases:
\begin{itemize}
    \item $\exists h=(b,c,\star)\in F_2$: in this case is enough to add $g$ to $F_2$ and remove $h$ from $F_2$, after this the value for $p_2$ of $F_2$ is increased by $b-c$ while remained constant for $p_1$, so we would have that $v_2(F_2)\ge a-b-c$ and $v_1(F_2) \le a-b$, so we can unfreeze $p_1$ and move forward. 
    \item $(c,b,\star)\in F_2$: in this case, by excluding the precedent case, we have that is impossible have that $v_2(F_2) \le a-b-c$ and $v_2(F_2) +v_2(g)\ge a-b$ and $v_1(F_2) \le a-b-c$ and $v_1(F_2) +v_1(g)\ge a-b$, because $v_2(F_2) = v_2(F_2\setminus \{g\}) + b$ and $v_1(F_2) = v_1(F_2\setminus \{g\}) + c$ and $v_2(F_2\setminus \{g\}) \ge v_1(F_2\setminus \{g\})$, so we have in the end that $v_2(F_2) - v_1(F_2) \ge b-c$
    \item only $(b,b,\star), (c,c,\star)\in F_2$: in this case we can remove the items that causes the shift between multiples of $b$ and the value of $F_2$ for $p_2$ and $p_1$ (in this case $v_1(F_2) = v_2(F_2)$). These items that we remove have value lower of $b$ for $p_1$ and $p_2$ and we can assign them to $p_3$ (and $p_3$ will not take the item that he would have taken in the last iteration in which we had assigned to $p_2$ the item $g$), than we assign to $F_2$ the item $g$. So now the value of $F_2$ will be lower or equal to $a-b$. 
\end{itemize}
In all the precedent cases we have removed an item $h$ from $F_2$ without considering the value for $p_3$ and so we could think "what if that item had value higher than the ones that already took $p_3$?" This gives no problem because we already had assumed that $p_3$ could leave an item like $g$ as last item to $p_1$ or $p_2$, so if we give such item to $p_2$ and than we assign him $f$ as last item this would make no difference to assign $h$ while $p_1$ was frozen and $g$ as last item, instead if we assign $h$ as last item to $p_1$, we know that $v_3(F_3)\ge v_3(h)$, so we still have an EFX allocation. 

\paragraph{After the problematic assignment} After such problematic assignment we have a case in which we can have another problematic assignment, such case is the one in which $p_3$ has still an item valued $a$ after that $p_1$ has been unfrozen. If this happens we have that since we start from having all players that are EF towards each other except for $p_2$ that has $v_2(F_2) + c > v_2(F_1) \ge v_2(F_2)$ we have that we can obtain an EFX allocation by following the rules for such second problematic assignment. Indeed each problematic assignment has been solved by leading to an EF allocation except for the unfrozen player who can still envy the frozen one for at most $c$, so in this case is enough that if we have to chose one player to freeze between $p_1$ and $p_2$, we chose to freeze $p_2$, avoiding to let him envy for more than $c$ $p_1$. We can do such a choice since if both $p1$ and $p2$ are involved in another problematic assignment than both have that the remaining items value only $c$.

\chapter{Conclusions}
In this thesis, I have shown that by starting from the idea provided in the \textit{Match\&Freeze} algorithm \cite{DBLP:MaximumNashWelfareandOtherStoriesAboutEFX} we can obtain an algorithm for obtaining an EFX allocation with two players and three values. Moreover, since the problem of EFX allocation for two players has already been solved with the divide and choose an algorithm, I have improved the approach for two players to deal with the case of three players and three values with the constraint that $c \ge a \mod b$. Before this thesis, the problem for finding EFX allocation for three players had only one result that proved that there is always an EFX allocation for three players. Such proof builds to a pseudo-polynomial algorithm to solve this problem. Instead, I have described an approach to obtain an EFX allocation for three players, three values with a constraint, in polynomial time. Such thesis can be further expanded in two main ways: 
\begin{itemize}
    \item deal with the case in which  $c > a \mod b$ does not hold,
    \item try to expand the work for four players.
\end{itemize}

While writing this thesis I mainly tried to solve the first of these two points. The main problem for which it is harder dealing with such a case is that for how I built the two players case I have that when we have a problematic assignment and the non frozen players can still take items with value $b$, they could have to take some items valued $c$ to respect the constraint for which the value of the set of items obtained while the envied player is frozen has to have a value between $a-b-c$ and $a-b$. So in the case of three players, we could have that the two non frozen players could start to envy each other. An approach that I tried and did not work with some rare cases, that maybe could be handled in a different manner, was to delay the non frozen players and let them take value while frozen by avoiding the part with the items valued $c$ while there were still items valued $b$ and then, when the frozen player takes the last item valued $b$ let the other two players take the needed value. Such an approach had problems when one of the two non frozen players finished the items valued $b$ before the frozen player.

\backmatter



% bibliography
%\cleardoublepage
%\phantomsection
%\bibliographystyle{sapthesis} % BibTeX style
%\bibliography{bibliography} % BibTeX database without .bib extension

\printbibliography

\end{document}
