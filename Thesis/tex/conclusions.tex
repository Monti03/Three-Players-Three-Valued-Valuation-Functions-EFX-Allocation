\chapter{Conclusions}
In this thesis, I have shown that by starting from the idea provided in the \textit{Match\&Freeze} algorithm \cite{DBLP:MaximumNashWelfareandOtherStoriesAboutEFX} we can obtain an algorithm for obtaining an EFX allocation with two players and three values. Moreover, since the problem of EFX allocation for two players has already been solved with the divide and choose an algorithm, I have improved the approach for two players to deal with the case of three players and three values with the constraint that $c \ge a \mod b$. Before this thesis, the problem for finding EFX allocation for three players had only one result that proved that there is always an EFX allocation for three players. Such proof builds to a pseudo-polynomial algorithm to solve this problem. Instead, I have described an approach to obtain an EFX allocation for three players, three values with a constraint, in polynomial time. Such thesis can be further expanded in two main ways: 
\begin{itemize}
    \item deal with the case in which  $c > a \mod b$ does not hold,
    \item try to expand the work for four players.
\end{itemize}

While writing this thesis I mainly tried to solve the first of these two points. The main problem for which it is harder dealing with such a case is that for how I built the two players case I have that when we have a problematic assignment and the non frozen players can still take items with value $b$, they could have to take some items valued $c$ to respect the constraint for which the value of the set of items obtained while the envied player is frozen has to have a value between $a-b-c$ and $a-b$. So in the case of three players, we could have that the two non frozen players could start to envy each other. An approach that I tried and did not work with some rare cases, that maybe could be handled in a different manner, was to delay the non frozen players and let them take value while frozen by avoiding the part with the items valued $c$ while there were still items valued $b$ and then, when the frozen player takes the last item valued $b$ let the other two players take the needed value. Such an approach had problems when one of the two non frozen players finished the items valued $b$ before the frozen player.