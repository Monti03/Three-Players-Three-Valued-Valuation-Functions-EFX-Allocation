\chapter{Three Players}
\section{Constraint on the Values}
In the next pages I am going to show how to obtain an EFX allocation with three players and three values with the constraint that $c\ge a\mod b$. This constraint was not present in the discussion over two players, so why do I add it now? The answer is in how we deal with the fact that the frozen player envied the non frozen one in the two player case: it was enough to invert the problematic assignment. With the three-player case, this is no longer possible: let's consider the problematic assignment in which $p_1$ is frozen, $p_2$ and $p_3$ envies him and still can take other items valued $b$, in such a case both $p_2$ and $p_3$ by following the approach described in the case of two players could have to take items valued $c$ before finishing the ones valued $b$ if $c< a\mod b$, so we could have that not only $p_1$ starts to envy the non-frozen players, but also that the non-frozen player starts to envy each other and this is not solvable by changing the problematic assignment.
\section{Redefinition of Problematic Assignment}
For two players we have that only when the frozen player values remaining items as $b$, we could have a non EFX allocation by following the maximum matching assignments because in the end the frozen player could envy the non frozen player. In order to see what happens with three players we have to consider the different cases in which the frozen player values all the remaining items as $c$, let's consider that during the assignments that leads to freeze player $p_1$ he took item $x$, $p_2$ took item $y$ and $p_3$ took item $z$: 
\begin{itemize}
    \item $v_1(x) = a, v_2(x) = a, v_2(y) = b, v_3(x) = a, v_3(z) = c$: in this case $p_3$ needs exactly $\floor{a-c}{c}$ items that have value $c$ for $p_1$ and $p_3$, so $p_1$ when unfrozen will not envy $p_3$ (we have to notice that $v_1(z) = c$ otherwise we would have assigned $x$ to $p_3$ and $z$ to $p_1$ since $a + b + b > a + b + c$). Instead $p_2$ needs in the worst case (in the case in which he values all the remaining items as $c$) $\floor{a-b}{c}$ items, that have value lower or equal to $a-b-c$ for $p_1$, so also in this case $p_1$ will not envy $p_2$.
    
    \item $v_1(x) = a, v_2(x) = b, v_2(y) = c, v_3(x) = b, v_3(z) = c$: in this case both $p_2$ and $p_3$ have to take $\floor{b-c}{c}$ items. We have to see if $\floor{b-c}{c}\le \floor{a-w}{c}$ holds where $w=\max\{v_1(y), v_1(z)\}$. If $w=c$, than surely the condition above holds, instead in the case of $w=b$ let's say that $v_1(y) = b$, than we have that $v_1(x) + v_2(y) \ge v_1(y) + v_2(x)$ because of the maximum matching rule, so $a + c \ge 2b$. So from the precedent constraint we can write that $a -b \ge b-c$, so we have that $\floor{a-b}{c} \ge \floor{b-c}{c}$. So in both cases, we have that the number of items obtained by the frozen player is lower to the ones needed to $p_1$ to envy one of the other players when unfrozen.
    
    \item $v_1(x) = b, v_2(x) = b, v_2(y) = c, v_3(x) = b, v_3(z) = c$: in this case the frozen player will take $\floor{b-c}{c}$ items, so $p_1$ will envy no player when unfrozen because $v_1(y) = v_1(z) = c$ for the maximum matching assignment.
\end{itemize}
\section{We Cannot Use Maximum Matching}
In the case of three players we cannot use the maximum matching technique to assign the items while a player is frozen in one of the problematic assignments because we cannot control the value for the frozen player of both the players. In table \ref{table:example-maximum-matching-problem} there is an example that shows an allocation in bold obtained by choosing the items with the maximum matching technique while $p_1$ is frozen. As we can see also if we invert the assignment by freezing $p_2$ we can obtain the same assignment that leads the frozen player to envy $p_3$. In this particular case, by considering $a=400, b=100$ and $c=40$, in the end we have that $v_1(A_1) = a + c = 440$ while $v_1(A_3) - \min_{x \in A_3} v_1(x) = 5b + c -c = 500$, so we obtain a non EFX allocation. 
\begin{table}[h]
\centering
\begin{tabular}{lllllllllllllll}
                            & $i_1$                           & $i_2$                           & $i_3$                           & $i_4$                           & $i_5$                           & $i_6$                           & $i_7$                           & $i_8$                           & $i_9$                           & $i_{10}$                        & $i_{11}$                        & $i_{12}$                        & $i_{13}$                        & $i_{14}$                        \\ \hline
\multicolumn{1}{|l|}{$p_1$} & \multicolumn{1}{l|}{\textbf{a}} & \multicolumn{1}{l|}{b}          & \multicolumn{1}{l|}{b}          & \multicolumn{1}{l|}{b}          & \multicolumn{1}{l|}{c}          & \multicolumn{1}{l|}{b}          & \multicolumn{1}{l|}{c}          & \multicolumn{1}{l|}{b}          & \multicolumn{1}{l|}{c}          & \multicolumn{1}{l|}{b}          & \multicolumn{1}{l|}{c}          & \multicolumn{1}{l|}{c}          & \multicolumn{1}{l|}{c}          & \multicolumn{1}{l|}{\textbf{c}} \\ \hline
\multicolumn{1}{|l|}{$p_2$} & \multicolumn{1}{l|}{a}          & \multicolumn{1}{l|}{\textbf{b}} & \multicolumn{1}{l|}{b}          & \multicolumn{1}{l|}{b}          & \multicolumn{1}{l|}{\textbf{b}} & \multicolumn{1}{l|}{b}          & \multicolumn{1}{l|}{\textbf{b}} & \multicolumn{1}{l|}{c}          & \multicolumn{1}{l|}{\textbf{c}} & \multicolumn{1}{l|}{c}          & \multicolumn{1}{l|}{\textbf{c}} & \multicolumn{1}{l|}{c}          & \multicolumn{1}{l|}{\textbf{c}} & \multicolumn{1}{l|}{c}          \\ \hline
\multicolumn{1}{|l|}{$p_3$} & \multicolumn{1}{l|}{b}          & \multicolumn{1}{l|}{b}          & \multicolumn{1}{l|}{\textbf{b}} & \multicolumn{1}{l|}{\textbf{b}} & \multicolumn{1}{l|}{b}          & \multicolumn{1}{l|}{\textbf{b}} & \multicolumn{1}{l|}{b}          & \multicolumn{1}{l|}{\textbf{b}} & \multicolumn{1}{l|}{c}          & \multicolumn{1}{l|}{\textbf{b}} & \multicolumn{1}{l|}{c}          & \multicolumn{1}{l|}{\textbf{c}} & \multicolumn{1}{l|}{c}          & \multicolumn{1}{l|}{c}          \\ \hline
\end{tabular}
\caption{An assignment that shows the problem of using maximum matching while a player is frozen because of a problematic assignment}
\label{table:example-maximum-matching-problem}
\end{table}

\section{First Problematic Assignment}
\begin{table}[h]
\centering
\begin{tabular}{|l|l|l|l||l|l|l|l|}
\hline
      &            &            &            & x & y & z & w \\ \hline
$p_1$ & \textbf{a} & b          & c          & b & b & c & c \\ \hline
$p_2$ & a          & \textbf{b} & c          & b & c & b & c \\ \hline
$p_3$ & b          & c          & \textbf{c} & c & c & c & c \\ \hline
\end{tabular}
\caption{}
\label{table-3-players-aab}
\end{table}
In the table \ref{table-3-players-aab} the letters above each column are used in order to represent the type and also to represent the number of such items . Let's consider that we assign the items in bold, than we have to consider different cases:
\begin{itemize}
    \item $w \ge \left \lfloor \frac{b}{c}\right \rfloor - 1$: in this case we can assign $\left \lfloor \frac{b}{c}\right \rfloor - 1$ items of type $w$ to $p_3$ so that he does no longer envies $p_1$. Now we have the following situation for $p_1$:
    \begin{align*}
        v_1(A_2) &= v_1(A_2^*) + b\\
        v_1(A_3) &= v_1(A_3^*)  +\left \lfloor \frac{b}{c}\right \rfloor c \le v_1(A_3^*)  + b \\
    \end{align*}
    so for $p_1$ we can threat the other two players in the same manner. Now we assign to $p_2$ and $p_3$ items of type $x$, $z$ and $y$. Since $c \ge a\%b$ $p_2$ needs only items of value $b$ to reach the constraint so we can consider the following cases:
    \begin{itemize}
        \item $ \left \lfloor \frac{x+z}{2}\right \rfloor \ge  \left \lfloor \frac{a-b}{b}\right \rfloor$: in this case is enough to let take the same number of items of each type to both $p_2$ and $p_3$ till reaching $\left \lfloor \frac{a-b}{b}\right \rfloor$ items. It is easy to see that $p_1$ will not envy $p_2$ or $p_3$ since $a-b\ge v_2(F_2) \ge v_1(F_2)$.
        \item in the case in which the condition above does not hold, than we have that $p_2$ and $p_3$ need to take $\hat y$ items of type $y$ and this is not a problem if
        \begin{equation*}
            \begin{cases}
                a -b \ge  \left \lfloor \frac{x+z}{2}\right \rfloor b + \hat{y}c\\
                a -b \ge  \left \lfloor \frac{x}{2} + 1\right \rfloor b +  \left \lfloor \frac{z}{2} - 1\right \rfloor c + \hat{y}b
            \end{cases}
        \end{equation*}
        So we have a non EFX allocation if $\hat{y} > z$, in this case is enough to invert the assignment to the one shown in table \ref{table-3-players-aab-invert-w-greater-second-case} since is like considering $p_1$ inverted with $p_2$, so exchanging $y$ with $z$.
        \begin{table}[h]
            \centering
                \begin{tabular}{|l|l|l|l|}
                \hline
                $p_1$ & a          & \textbf{b} & c           \\ \hline
                $p_2$ & \textbf{a} & b          & c           \\ \hline
                $p_3$ & b          & c          & \textbf{c}  \\ \hline
                \end{tabular}
            \caption{}
            \label{table-3-players-aab-invert-w-greater-second-case}
        \end{table}
    \end{itemize}
    
    \item $w < \left \lfloor \frac{b}{c}\right \rfloor - 1 $: in this case we will not assign all the items of type $w$ to $p_3$. So in this case we have that $p_1$ envies no one, $p_2$ envies only $p_1$ and the same $p_3$. In order to remove the fact that $p_2$ and $p_3$ still envy $p_1$ we need to assign to $p_2$ $\left \lfloor \frac{a-b}{b} \right \rfloor $ items of type $x$ or $z$ and other $\left \lfloor \frac{a-b - \floor{x+z}{2}b}{c} \right \rfloor$ items of type $y$ or $w$ if the items of type $x$ and $z$ are not enough; instead $p_3$ needs to obtain $\left\lfloor \frac{b}{c}\right \rfloor -1$ items of any type. So let's consider the following cases:
    \begin{itemize}
        \item $\left \lfloor \frac{x+z}{2} \right \rfloor\ge \left \lfloor \frac{a-b}{b} \right \rfloor$ and $\left \lfloor \frac{x+z}{2} \right \rfloor\ge \left\lfloor \frac{b}{c}\right \rfloor -1 $: in this case by assigning to $p_2$ and $p_3$ the items of type $x$ and $z$ till $p_2$ does not envy $p_1$ leads to an EFX allocation cause for $p_1$ the value obtained by the two players is lower or equal to $a-b$ as for $p_2$, for $p_2$ the obtained values is larger or equal to $a-b-c$ and for $p_3$ we have obtained the required number of items. 
         \item $\left \lfloor \frac{x+z}{2} \right \rfloor\ge \left \lfloor \frac{a-b}{b} \right \rfloor$ and $\left \lfloor \frac{x+z}{2} \right \rfloor< \left\lfloor \frac{b}{c}\right \rfloor -1 $: in this case we do not always obtain an EFX allocation because in order to have $p_3$ to not envy $p_1$ we could have to give to $p_3$ too many items that have value $b$ for $p_1$, so $p_1$ would envy him when unfrozen. 
         %In this case we can invert the initial assignment to the one shown in table 
         We can consider the following two case:
         \begin{itemize}
            \item $y\ge \floor{z'}{2}$: in this case we can change the assignment to the one show in table \ref{table-3-players-aab-invert-w-lower-second-case-p3p1p2}. Than we can assign the items as follows: assign half of the $x$ items to $p_1$ and half to $p_2$ and than assign $\frac{z'}{2}$ items of type $z$ to $p_2$ and $\frac{z'}{2}$ items of type $y$ to $p_1$. If $x$ is odd, than also $z'$ is, in this case we have to assign the $x$ odd item to $p_1$ and the $z'$ odd item to $p_2$ in order to give each player $\floor{a-b}{b}$ items that he values $b$. So we have assigned $\floor{x+z'}{2}$ items to each player and now both no longer envy $p_3$, moreover $p_3$ will not envy them since $\floor{x+z'}{2} < \floor{b}{c} - 1$.
            \item $y < \floor{z'}{2}$: in this case we can avoid to change the assignment since, if $y< \frac{z'}{2}$ than $y < \left \lfloor \frac{a-b-\frac{x}{2}}{b}\right \rfloor$.
             %, so also if $p_2$ and $p_3$ take all the items of type $y$, than the value obtained by the two players for $p_1$ will still be lower than $a-b$, so $p_1$ will not envy any of the other two players. 
            In this case we have to do the following assignment: assign to $p_2$ $\frac{x}{2}$ items of type $x$, $y$ items of type $z$ and than $\frac{z'}{2}-y + \floor{b-c-\frac{x}{2}c - \frac{z'}{2}c}{c}$ items of type $z$, instead we have to assign to $p_3$: $\frac{x}{2}$ items of type $x$, $y$ items of type $y$ and than $\frac{z'}{2}-y + \floor{b-c-\frac{x}{2}c - \frac{z'}{2}c}{c}$ items of type $z$. As we can see $p_1$ will not envy $p_2$ since 
            \begin{align*}
                v_1(F_2) &= \frac{x}{2}b + \frac{z'}{2}c + \floor{b-c-\frac{x}{2}c - \frac{z'}{2}c}{c} c \\
                & \le a-2b + b - c = a- b - c
            \end{align*}    
            since we are considering that $y<\floor{z'}{2}$, so $\floor{z'}{2} \ge 1$ we have $ \frac{x}{2}b \le a-2b$ and we also have that $\frac{z'}{2}c + \floor{b-c-\frac{x}{2}c - \frac{z'}{2}c}{c} c \le b-c$ . 
            $p_1$ will also not envy $p_3$ as we can see by the next equation:
            \begin{align*}
                v_1(F_3) &= \frac{x}{2}b + yb + (\frac{z'}{2}-y)c \floor{b-c-\frac{x}{2}c - \frac{z'}{2}c}{c} c \\
                & \le a-2b + b - c = a- b - c
            \end{align*}
            since, as before,  $y<\frac{z'}{2}$, we have $\frac{x}{2}b + yb\le a-2b$ and we also have that $(\frac{z'}{2}-y)c \floor{b-c-\frac{x}{2}c - \frac{z'}{2}c}{c} c \le b-c$.
            We can notice that if $x$ is odd, than we still have the same conditions if we consider that we give the $x$ odd item to $p_3$ and that $y < \floor{a-b}{b} - \floor{x}{2} - 1$, if this does not hold, than we can see that we obtain an EFX allocation by following the approach described in the precedent case.
         \end{itemize}
         
         \begin{table}[h]
            \centering
                \begin{tabular}{|l|l|l|l|}
                \hline
                $p_1$ & a               & \textbf{b}    & c             \\ \hline
                $p_2$ & a               & b             & \textbf{b}    \\ \hline
                $p_3$ & \textbf{b}      & c             & c             \\ \hline
                \end{tabular}
            \caption{}
            \label{table-3-players-aab-invert-w-lower-second-case-p3p1p2}
        \end{table}
        \item $\floor{x + z}{2} < \floor{a-b}{b}$: in this case we must take also items of type $y$. Let's consider that we give $y'$ items to the non frozen players of type $y$, than we must have that two or none between $x$, $y'$ and $z$ are odd because $x+y'+z$ has to be even in order to split the items between the two non frozen players. Let's deal with the possible case one per time:
        
        \begin{itemize}
            \item $\floor{y'}{2} \le \floor{z}{2}$: in order to solve this problem we have to do the following assignments: to $p_2$ and $p_3$ $\floor{x}{2}$ items of type $x$, than we assign one item of the two types that have an odd number of items by giving the one (there is always at least one) with value $b$ for $p_2$ to $p_2$, than we assign to each player $\floor{z}{2}$ items of type $z$ and in the end we assign $\floor{y'}{2}$ items of type $y$ to each player. In all cases, since the items of type $z$ given to each player are more than the ones of type $y$, the value for $p_1$ of the value obtained by the two players will be lower or equal to the value obtained by $p_2$ that is lower or equal to $a-b$.
            \item $\floor{y'}{2} > \floor{z}{2}$: we can see that in this case we can do the same thing done before by exchanging $p_1$, $p_2$ and $z$ and $y'$.
        \end{itemize}
        In all the above cases I have assumed that the number of items obtained by $p_3$ are greater or equal to $\floor{b}{c} -1$ because when we assign to $p_2$ items with value $c$ for him (items of type $y$ and $w$), we are assigning them in order give him at least a value of $b-c$ (because we have the constraint $c\ge a \mod b $), so also considering only these items, $p_3$ took enough items to no longer envy $p_1$.
    \end{itemize}
 
 \end{itemize}
  We can notice that for how we give the items to the two non frozen players, we will never have that one player envies the other, so we will never have to freeze a player in this phase. Instead after that the frozen player $p_i$ is unfrozen, we could have another assignment that can lead to a freeze: in this case we can see that $p_3$ is not a cause of this freeze since all the remaining items have value $c$, so we will freeze one player among $p_1$ and $p_2$. If this happens, than we have to freeze the player that has not yet been frozen. Let's consider that in the first freeze we froze $p_1$, than when $p_1$ is unfrozen he is envy-free to the other two players, while for $p_2$ we have that $v_2(A_2) \ge v_2(A_1) - c$, so by freezing $p_2$ we ensure the fact that $p_1$ takes value lower than $p_2$ with a difference of at most a $c$, while $p_2$ takes items with larger or equal value to the ones obtained by $p_1$, so we are ensuring the fact that $p_1$ and $p_2$ are EFX.

\section{Second Problematic Assignment}
\begin{table}[h]
\centering
\begin{tabular}{|l|l|l|l||l|l|}
\hline
      & $i_1$           & $i_2$      & $i_3$        & z & w \\ \hline
$p_1$ & \textbf{a}      & b          & b            & b & c \\ \hline
$p_2$ & b               & \textbf{c} & c            & c & c \\ \hline
$p_3$ & b               & c          & \textbf{c}   & c & c \\ \hline
\end{tabular}
\caption{}
\label{table-3-players-abb-bcc-bcc}
\end{table}
In this case is easy to see that $p_2$ and $p_3$ both need $\floor{b-c}{c}$ items of any remaining type. We can differentiate two cases by considering the number $\hat z$ of items of type $z$ that we have to assign to $p_2$ and $p_3$ to reach the required number of items
\begin{itemize}
    \item $\floor{w}{2}c + \hat z b \le a-b$: where $\hat z = \floor{b-c-\floor{w}{2}c}{c}$, in this case is enough to assign to $p_2$ and $p_3$ first the items of type $w$, than the items of type $z$ in equal values.
    \item $\floor{w}{2}c + \hat z b > a-b$: where $\hat z = \floor{b-c-\floor{w}{2}c}{c}$, in this case we have to invert the assignment so that we freeze player $p_2$ rather than $p_1$. Now we have that $p_1$ and $p_3$ will envy $p_2$ for different values. By assigning the items of type $z$ to $p_1$ and of type $w$ to $p_3$ till $p_3$ reaches $\floor{b-c}{c}$ items, we have an EFX allocation since 
    \begin{itemize}
        \item $p_1$ will not envy $p_2$ since he surely took more than $a-b$ value and equal or higher value than $p_3$ while $p_2$ was frozen,
        \item $p_2$ will not envy the other players since they took at most $b-c$ value while $p_2$ was frozen,
        \item $p_3$ will not envy $p_1$ since they take the same items and will not envy $p_2$ by considering that $p_3$ ha precedence over the last iteration assignment.
    \end{itemize}
\end{itemize}


\section{Third Problematic Assignment}

\begin{table}[h]
\centering
\begin{tabular}{|l|l|l|l||l|l|l|l|l|l|l|l|}
\hline
      &                 &               &               & x & y & z & j & k & l & w & v \\ \hline
$p_1$ & \textbf{a}     & b             & b             & b & c & b & b & c & b & c & c \\ \hline
$p_2$ & a               & \textbf{b}    & b             & b & b & c & b & b & c & c & c \\ \hline
$p_3$ & a               & b             & \textbf{b}    & c & c & c & b & b & b & c & b \\ \hline
\end{tabular}
\end{table}
Let's assume that 
$$x + y + k + j \ge j + k + l + v \ge x + z + j + l$$
so that the second player has the higher number of items that he values $b$, followed by $p_3$ and $p_1$. Now we can show that $p_1$ will never envy the other two players when he is unfrozen if we assign the items as follows till both player have reached $v_i(F_i) \ge a-b-c$: we assign first the items of type $y$ and $x$ to $p_2$, and the items of type $v$ and $l$ to $p_3$. Since we have assumed that $x + y + k + j \ge j + k + l + v $, we have $x + y  \ge  l + v $ so $p_3$ will finish first the items of type $l$ and $v$, so while $p_2$ is still taking items of type $x$ and $y$ he will star taking items of type $k$ and $l$. Now we can have two different cases:
\begin{itemize}
    \item $x + y \ge l + v + j + k $: in this case $p_3$ will finish all the items that he values $b$ ($l,v,j$ and $k$) while $p_2$ is still taking items of type $x$ and $y$. So after that $p_3$ has finished these items we will start assigning the same type of items to the two non frozen players in the following order $x$,$y$, $w$ and $z$. In this case is easy to see that $p_1$ will not envy the non frozen players when he will be unfrozen because $p_3$ takes all the items that he values $b$ and by the initial assumption we have that $j + k + l + v \ge x + z + j + l$, so $p_3$ will surely achieve $a-b-c$ before $p_1$ does it considering the bundle obtained by one of the two non frozen players.
    \item $x + y < l + v + j + k $: in this case $p_2$ will finish all the items of type $x$ and $y$ when there are still left other items of type $j$ and $k$. We define to be $k'$ and $j'$ the items of type $k$ and $j$ respectively taken by $p_3$ while $p_2$ is still taking items of type $x$ and $y$, and we define $rem_k$ and $rem_j$ to be $k-k'$ and $j-j'$ respectively. We will divide $rem_k$ and $rem_j$ equally among the two non frozen players and if the remaining $j$ and $k$ items are odd, we give that item to $p_2$ and we freeze him till $p_3$ obtains $\floor{b-c}{c}$ items among the remaining ones that are of type $w$ and $z$. 
    Both in case of a second freeze and in case of no second freeze while $p_1$ is frozen, we have that at most $p_1$ will envy one of the two other players and not both. In table \ref{table:abb-abb-abb-assignment-in-the-case-of-second-freeze} and  \ref{table:abb-abb-abb-assignment-in-the-case-of-no-second-freeze} we can see the assignments done for the case in which we have a freeze and the case in which we have no freeze, where we consider $rem_z$ to be the number of remaining items of type $z$ after that $p_2$ has been unfrozen in the case of second freeze.
    We can see that for $p_1$ to envy both other players we must have that both $v_1(F_2) > v_2(F_2)$ and $v_1(F_3) > v_3(F_3)$ are true. In the following equations we can see that this is impossible in the case of second freeze while $p_1$ is frozen, but this can also consider the case in which we have no second freeze because is enough to have $rem_z = z$.
    
    \begin{align*}
        v_1(F_2) &= xb + yc + \floor{rem_j}{2}b + \floor{rem_k}{2}c + b + \floor{rem_z}{2}b + \floor{rem_w}{2}c\\
        v_2(F_2) &= xb + yb + \floor{rem_j}{2}b + \floor{rem_k}{2}b + b + \floor{rem_z}{2}c + \floor{rem_w}{2}c\\
        \\
        v_1(F_3) &= vc + lb + j'b + k'c + \floor{rem_j}{2}b + \floor{rem_k}{2}c + c + (z-rem_z)b \floor{rem_z}{2}b + \floor{rem_w}{2}c\\
        v_3(F_3) &= vb + lb + j'b + k'b + \floor{rem_j}{2}b + \floor{rem_k}{2}b + c + (z-rem_z)c \floor{rem_z}{2}c + \floor{rem_w}{2}c
    \end{align*}
       
    \begin{align*}
        &v_1(F_2) > v_2(F_2)\implies \floor{rem_z}{2} > y + \floor{rem_k}{2} \\
        &v_1(F_3) > v_3(F_3)\implies \floor{rem_z}{2} + z-rem_z > v + k - rem_k+  \floor{rem_k}{2} 
    \end{align*}
    By summing the two inequalities we obtain
    \begin{align*}
        &2\floor{rem_z}{2} + z-rem_z > y + v + k - rem_k+  2\floor{rem_k}{2} 
    \end{align*}
    that is impossible because of the initial assumption, indeed $z \le k + v + y$.
    So $p_1$ will at most envy one of the other two players. This will happen only when the two non frozen player are not able to reach $a-b-c$ with the only items of type $b$ because otherwise is immediate that the value for $p_1$ of the bundles obtained by the other two players while he was frozen is lower or equal to the value obtained by the two non frozen players that is lower or equal to $a-b-c$. So in the case in which $p_1$ envies one of the two other players we can swap the bundles obtained from the problematic iteration till the iteration in which we unfreeze $p_1$ of these two players.
    %Let's start considering the case in which we have a second freeze, than we have assigned to $p_2$ the following items $x$, $y$, $\floor{rem_k}{2}$ of type $k$, $\floor{rem_j}{2}$ of type $j$, the odd item (type $k$ or $j$ that provokes the second freeze), $\floor{z'}{2}$ of type $z$ and $\floor{w}{2}$ of type $w$; instead $p_3$ takes $v$, $l$, $x + y - v- l$ of type $j$ and $k$,  $\floor{rem_k}{2}$ of type $k$, $\floor{rem_j}{2}$ of type $j$ an item of type $z$ or $w$, $z-z'$  items of type $z$ and the items of type $w$ while $p_2$ is frozen 
    
    \begin{table}[h]
        \begin{tabular}{|l|l|l|l|l|l|l|l|l|l|l|l|}
            \hline
            $p_2$ & $x$ & $y$ &      &    &   $\floor{rem_k}{2}$ & $\floor{rem_j}{2}$ & odd k or j & frozen & $\floor{rem_w}{2}$ & $\floor{rem_z}{2}$ & z odd \\ \hline
            $p_3$ & $v$ & $l$ & $k'$ & $j'$ & $\floor{rem_k}{2}$ & $\floor{rem_j}{2}$ & w or z     & w, z   & $\floor{rem_w}{2}$ & $\floor{rem_z}{2}$ & w odd \\ \hline
        \end{tabular}
        \caption{Assignment in the case of second freeze while $p_1$ is frozen}
        \label{table:abb-abb-abb-assignment-in-the-case-of-second-freeze}
    \end{table}
    
    \begin{table}[h]
        \begin{tabular}{|l|l|l|l|l|l|l|l|l|l|}
            \hline
            $p_2$ & $x$ & $y$ &      &    &   $\floor{rem_k}{2}$ & $\floor{rem_j}{2}$ & $\floor{w}{2}$ & $\floor{z}{2}$ & z odd \\ \hline
            $p_3$ & $v$ & $l$ & $k'$ & $j'$ & $\floor{rem_k}{2}$ & $\floor{rem_j}{2}$ & $\floor{w}{2}$ & $\floor{z}{2}$ & w odd \\ \hline
        \end{tabular}
        \caption{Assignment in the case of no second freeze while $p_1$ is frozen}
        \label{table:abb-abb-abb-assignment-in-the-case-of-no-second-freeze}
    \end{table}
    
\end{itemize}

Now let's consider first the case in which we had no swap after the unfreeze of $p_1$ and than the case in which we have it:

\begin{itemize}
    \item in the cases in which we had no swap after the unfreeze of $p_1$ we have the following situation 
    \begin{align*}
    &v_1(\hat A_1) + a \ge v_1(\hat A_2) + b + v_1(F_2)\\
    &v_1(\hat A_1) + a \ge v_1(\hat A_3) + b + v_1(F_3)\\
    \\
    &v_2(\hat A_2) + b + v_2(F_2) \ge v_2(\hat A_1) + a - c \\
    &v_2(\hat A_2) + b + v_2(F_2) \ge v_2(\hat A_3) + b + v_2(F_3)\\
    \\
    &v_3(\hat A_3) + b + v_3(F_3) \ge v_3(\hat A_1) + a - c\\
    &v_3(\hat A_3) + b + v_3(F_3) \ge v_3(\hat A_2) + b + v_3(F_2) - c\\
    \end{align*}
Where in the last inequality we have the last $-c$ only if we had a second freeze.

After this we have that in the last iteration we will assign the items with priority to $p_2$ and $p_3$, and among these two players with priority to the player who values more the item if we had no second freeze, or to $p_3$ if we had a second freeze for which we froze $p_2$. The precedent assertion works only when the last item has value $c$ for $p_1$, since he is EF towards the other two players till the last iteration, or $p_2$ and $p_3$ did not took any item of type $z$ while frozen (so in the case in which the last item is of type $j, k, x, y, l$ and $v$) since in this case each item obtained by the frozen players while $p_1$ was frozen, has value higher or equal, for the player who took it, to the value that has for $p_1$, so in the case in which the last item values $b$ for $p_1$ we can have two cases: $\min_{x\in A_i} v_1(x) = b$ or $\min_{x\in A_i} v_1(x) = c$, the first case is easy for the definition of EFX, the second case instead has more to argue. In the case in which  $\min_{x\in A_i} v_1(x) = c$, we have that or this item has been obtained while $p_1$ was frozen, and this would imply that $v_1(F_i) \le a - b - b + c = a-2b+c$ so we can assign the last item to player $i$ keeping $p_1$ EFX towards $p_i$, or is associated to an item that $p_1$ took with value $b$, so we can still assign the last item to $p_i$ by keeping $p_1$ EFX towards $p_i$. Instead if the last item is of type $z$, than we could have assigned some items of type $z$ to $p_i$, in this case we can't assert the same things that I have told before for the case in which $\min_{x\in A_i} v_1(x) = c$.  Let's divide the cases based on the values of each player for it's set of items obtained from the iteration in which we freeze $p_1$ and let's call this set for a generic player $p_i$ as $\bar{A_i}$: 
\begin{itemize}
    \item $v_2(\bar A_2) \ge v_2(\bar A_1)$ and $v_3(\bar A_3) \ge v_3(\bar A_1)$: in this case we have to assign the last item to $p_1$ and if the last items are two, if there was a second freeze, to the non frozen player, otherwise to one between $p_2$ and $p_3$. We can notice that since the item is valued $c$ by $p_2$ and $p_3$ we obtain an EFX allocation.
    \item $v_2(\bar A_2) \ge v_2(\bar A_1)$ and $v_3(\bar A_3) \ge v_3(\bar A_1) - c$: in this case we can notice that since $p_2$ took more value than the value obtained by $p_1$ or he had to wait $p_3$ ($x+y\ge j+k+v+l$) or after the freeze he took some item with value $b$ while in the same iteration $p_1$ took an item valued $c$ by $p_2$; in both cases we have that there has not been a second freeze between $p_2$ and $p_3$, so we have that $p_2$ and $p_3$ are EF till now. If there are two last items, we assign them to $p_3$ and $p_1$ since $p_2$ is EF towards both of them, instead if is only one item and we have that $v_1(\bar A_3) + b - c \le v_1(\bar A_1)$, we can assign the last item to $p_3$. The last case to consider is when we have only one last item and $v_1(\bar A_3) + b - c > v_1(\bar A_1)$, in this case we can swap the bundles of $p_1$ and $p_3$ and assign the last item to $p_1$. This because we will obtain $N_1 = \bar A_3, N_2 = \bar A_2$ and $N_3 = \bar A_1$ such that $v_1(N_1) + b - c\ge v_1(N_3)$ and $v_3(N_3) \ge v_3(N_1)$.
    \item $v_2(\bar A_2) \ge v_2(\bar A_1) - c$ and $v_3(\bar A_3) \ge v_3(\bar A_1)$: in this case hold the same things as before.
    \item $v_2(\bar A_2) \ge v_2(\bar A_1) - c$ and $v_3(\bar A_3) \ge v_3(\bar A_1) - c$: this case can be handled like the precedent one, but in this case we have that $p_2$ and $p_3$ could not be EF. If there is only one last item and we have that for at least one between $p_2$ and $p_3$ we have that $v_1(\bar A_i) + b - c \le v_1(\bar A_1)$ than we assign the last item to $p_i$. If instead this does not hold, than we can exchange the bundles of $p_1$ and $p_i$ where $i = \argmax_{i\in \{2,3\}} v_1(\bar A_i)$. Than we can assign the last items to $p_1$ and $p_j$ (not $p_i$). Assigning the last item also to $p_j$ could provoke to not having $p_i$ EFX towards $p_j$ because of a second freeze in which we froze $p_j$, so in this case we change that assignment by freezing $p_i$ and not $p_j$. We can notice that the value $v_1(\bar A_i)$ will only increase with this last swap, because $p_i$ will take more items than before, so the exchange of bundles between $p_1$ and $p_i$ done before will not change. 
\end{itemize}
    \item Instead in the case in which we had to swap after the unfreeze of $p_1$ we have that
\begin{align*}
    &v_1(\hat A_1) + b + v_1(F_2) \ge v_1(\hat A_2) + a\\
    &v_1(\hat A_1) + b + v_1(F_2) \ge v_1(\hat A_3) + b + v_1(F_3)\\
    \\
    &v_2(\hat A_2) + a \ge v_2(\hat A_1) + b + v_2(F_2) \\
    &v_2(\hat A_2) + a \ge v_2(\hat A_3) + b + v_2(F_3)\\
    \\
    &v_3(\hat A_3) + b + v_3(F_3) \ge v_3(\hat A_1) + b + v_3(F_2) - c\\
    &v_3(\hat A_3) + b + v_3(F_3) \ge v_3(\hat A_2) + a - c\\
\end{align*}

where the $-c$ in the last two rows is the first one because of in the case of second freeze while $p_1$ was frozen, we have freeze $p_2$ but now that bundle is of $p_1$ and the second one because $p_2$ now has the item valued $a$ and $p_3$ has that $v_3(F_3) \ge a - b - c$. In this case we have that the precedence to take the last items is of $p_3$ because is the only non EF player towards the others. 

In this case we can notice that $p_1$ and $p_2$ are EF towards the other two players and also that this case only happens when the two non frozen player start to take items of type $z$, so we can assert that the only two types of items that will be as last item are $w$ and $z$. In the case of last item of type $w$ is enough to assign it with precedence to $p_3$, instead in case of last item of type $z$, if there are two of such items, we can assign one to $p_1$ and one to $p_3$. Instead in the case of only one last item of type $z$ we have to deal with the fact that $p_3$ needs to take the last item and also that that item values $b$ for $p_1$. In the case in which in $F_2$ there is an item valued $c$ by $p_1$, we exchange that item with the last item and than we give it to $p_3$. Now I'll show that the bundle obtained while $p_1$ was frozen by the player who exchanged it with him has at least one item that $p_1$ values $c$: let's start considering $F_2$, than if by absurd we consider that $rem_k = 0$ than we must have that $x + y \ge v + l + k$, but than also $y = 0$, so we would have that $x \ge v + l + k$ and this violates the first assumption that for which $k + y \ge z + l$ and so $k \ge z + l$ and also $k + v \ge x + z$, but before we wrote that $x \ge v + l + k$. Instead in the case of $F_3$ we have that to have no item valued $c$, we must have that $k = 0, v = 0$ and this violates the initial assumption for which $k + v \ge x + z$.
\end{itemize}



\section{Problematic Assignments Where There Is a Player That Does Not Envy the Other}
\begin{table}[h]
\centering
\begin{tabular}{|l|l|l|l|l|l|l|l|l|l|l|l|l|}
\hline
      & d & e & f & g & j & k & y & x & v & l & z & w \\ \hline
$p_1$ & b & b & c & c & b & c & c & b & c & b & b & c \\ \hline
$p_2$ & b & c & b & c & b & b & b & b & c & c & c & c \\ \hline
$p_3$ & a & a & a & a & b & b & c & c & b & b & c & c \\ \hline
\end{tabular}
\caption{Types of items after a problematic assignment}
\end{table}
In this section I am going to describe the approach used to obtain EFX allocation when we have a problematic assignment in which a player ($p_3$ without loss of generality) does not envy the item obtained by the other two players. In the beginning of the proof I will show the case in which each type of item has an even number of elements, than I will show that in the case in which this not hold, we can obtain the same characteristics with other assignments. 
\paragraph{Even number of items} Let's consider the case in which we have even number of items for each remaining type after the problematic assignment, in this case we can assign the items to $p_3$ by letting him choose the one with higher value for him and than assign the same type of item to the other non frozen player ($p_2$ without loss of generality). By doing so we have that $v_1(F_2) = v_1(F_3)$, so if when unfrozen $p_1$ envies $p_3$ than he envies also $p_2$. We have two cases now:
\begin{itemize}
\item when unfrozen $v_1(F_2) \le a-b$: in this case we have that all players are EF towards each other except for $p_2$ that has $v_2(\bar A_1) \ge v_2(\bar A_2)> v_2(\bar A_1) - c$, where $\bar A_i$ is the set of items obtained by player $i$ from the problematic assignment on.
\item when unfrozen $v_1(F_2) > a-b$: in this case we have to invert the assignment and give $\bar A_1$ to $p_2$ and $\bar A_2$ to $p_1$ and after this we have that all the players are EF towards each other.
\end{itemize}
Now we have to deal with the fact that the player who has $F_2$ ($p_2$ in the first case and $p_1$ in the second case) did not took the items by first taking all the items with value $b$ and than the ones with value $c$ because he took always the same item took by $p_3$, so we could have some problems when assigning the last items. As first thing we can notice that we can avoid to give the last item to $p_3$ because he is EF towards $p_2$, $p_1$  and took the items in the right order, so we have to deal only on the case in which we have a single last item that we have to give to $p_1$ or $p_2$. The main problem raises when the last item is valued by both $p_1$ and $p_2$ as $b$. Now in order to show how to deal with this case I am going to consider the case in which there was no swap after that $p_1$ has been unfrozen, in the other case is enough to exchange $p_1$ with $p_2$. So if we have that $\min_{x\in A_2} v_1(x) = b$ than we can assign the item to $p_2$ and obtain an EFX allocation directly, instead in the case in which $\min_{x\in A_2} v_1(x) = c$ than we have to consider when the item $g$ valued $c$ by $p_1$ has been obtained: if it has been obtained before or after the problematic assignment, than surely $p_1$ took in the same iteration an item valued $b$ or higher, so by assigning the last item to $p_2$ we still have an EFX allocation. In the case in which the item valued $c$ by $p_1$ has been obtained when $p_1$ was frozen, than we have that the last item can be of two types: 
\begin{itemize}
    \item $g = (c,b,\star)$: in this case or we have that $\exists f=(b,c,\star)\in F_2$ or $v_1(F_2) \le a-2b + c$. In the latter case we can assign the last item directly to $p_2$ by keeping an EFX allocation, in the first case instead we can assign to $p_1$ $f$, while to $p_2$ we assign $g$. In this manner we decrease the value for $p_1$ and $p_2$ of the other player set of items of $b-c$, so we can assign to $p_2$ the last item. We can also notice that the value of $A_2$ will not increase for $p_3$ since the last item has surely lower or equal value to the item removed. Instead for what concerns $p_1$ for $p_3$, since $p_3$ did not envied him when $p_1$ was frozen, $p_3$ has $v_3(A_2)\le v_3(A_3\setminus F_3)$ and $v_3(F_3)\ge v_3(f)$, so also for $p_3$ this is still an EFX allocation.
    \item $g = (c,c,\star)$: this last case is the case in which all the items in $A_2$ valued $c$ by $p_1$ value also for him $c$ and have been obtained while $p_1$ was frozen. In this case we give all (or at most $\floor{b}{c}$) such type of items to $p_1$ from $F_2$ and assign the last item to $p_2$. If there are not enough such items we are still assigning them to $p_1$ and than we obtain that $\min_{x \in A_2} v_1(x) = b $, so we can assign the last item to $p_2$. If instead there are such items, we obtain an EFX allocation because the value for $p_1$ of $A_2$ increases at most of $c$. Also in this case for $p_3$ this is not a problem because the items assigned to $p_1$ from $F_2$ have still value lower or equal to $v_3(F_3)$.
\end{itemize}
\paragraph{non even number of items} in the precedent paragraph I have shown that by giving the same items to $p_3$ and $p_2$ chosen by $p_3$ we obtain an EFX allocation in the end, the main characteristic of such assignments that lead to an EFX allocation is that $v_1(F_2) \ge v_1(F_3)$ and that $p_2$ and $p_3$ are EF. In this paragraph I am going to show ho to keep such constraints when we have no even number of items and, in the case in which is not possible, how to still obtain an EFX allocation. In the next part I am going to show how to deal with the most difficult cases, for all the ones that have not been discussed here, it means that or we can assign the odd $g$ item to $p_3$ and we have another one $f$ such that $v_2(f)\ge v_2(g)$ and $v_1(f) \le v_1(g)$ or the opposite.
\begin{itemize}
    \item $d$ is odd: \begin{itemize}
        \item $e,f,g,j,x=0$: in this case we can assign to $p_2$ one item between $k$ or $z$ and one between $k$ or $y$, while to $p_3$ we assign the odd $d$ item. In this case I have to highlight that $p_3$ will not envy $p_2$ because is impossible to have $2b\ge a$, indeed in this case we would have that no item is taken while $p_1$ is frozen because $b\ge a-b$.
        \item $e,f,g,j,x, l, z=0$: in this case we have to make a manual assignments and not follow the algorithm described in this subsection. In this case we have that the only types of items that are left are $d,k,y,w$ and $v$. In this case we can split the $d$ items between $p_2$ and $p_3$ and than assign the odd $d$ item to $p_3$, while one between $k$ or $y$ to $p_2$ (or in the case in which $k=y=0$ we assign $\floor{b}{c}$ items between $w$ and $v$ and in the case in which these item have value higher of $a$ for $p_3$, than invert such last assignment)
        \item $e,f,g,j,x,k,y=0$: in this case, since $f+k=0$ we have that also $l+e=0$ and so $l=e=0$. So in this case we only have the following items: $d,z,w$ and $v$ moreover, since this case has to differ from the precedent one, we have that $z\ne 0$ (before we have also considered the case in which $l=z=k=y=0$) so in this case we can freeze $p_2$ and not $p_1$ and assign to $p_1$ and $p_3$ each half of the items of type $d$; than we assign to $p_3$ the odd $d$ item and to $p_1$ an item of type $z$, than we continue to assign items of type $z$ to $p_2$ and of type $v$ to $p_3$. When we unfreeze $p_2$, he will not envy the other two players.
    \end{itemize}
    \item $e$ is odd:
    \begin{itemize}
        \item $z,x,l,g,j=0$ in this case we can assign to $p_2$ the odd $e$ item and an item of type $y$, while to $p_3$ we assign an item of type $f$ and one of type $v$ or $w$
        \item also $y=0$: in this case we assign to $p_2$ the odd $e$ item and an item of type $k$, while to $p_3$ we have to assign an item of type $f$ and one of type $v$ (not $w$).
        \item also $v = 0$: in this case we cannot follow the algorithm, but in this case we have very few types of items, so we can assign the items manually as follows: we start assigning to $p_2$ the items of type $k$ (if there are, $k$ in this case could be also $0$) and $f$, while to $p_3$ we assign the items of type $e$ and since we are considering that $f+k\ge l+e$ and that $l=0$, than $k+f\ge e$ and so the $v_1(F_2)\le v_1(F_3) \le v_2(F_2)\le a-b$. So when unfrozen $p_1$ will not envy $p_2$ or $p_3$. From this we can easily obtain an EFX allocation as described in the beginning of the subsection. 
        \item $z,x,l,g,j,f=0$: in this case we have to assign the items manually in the following manner: first we assign the items of type $d$ to both $p_2$ and $p_3$, than, since $f+k\ge l+e$ and so $k\ge e$, we assign the items of type $k$ to $p_2$ and the items of type $e$ to $p_3$, than we will assign the remaining items of type $y$ and $v$ respectively to $p_2$ and $p_3$. As before, we have that $v_1(F_2)\le v_1(F_3) \le v_2(F_2)\le a-b$ and so we can obtain an EFX allocation in the end.
    \end{itemize}
    \item $l,v,k=0$ and $j$ is odd: in this case we can assign the odd $j$ item to $p_3$ and $\floor{b}{c}$ items between the ones of type $x,z,w,y$ to $p_2$. In this case we can notice that $p_1$ will envy more $p_3$ but we still have that $v_1(F_2) > v_1(F_3) - c$, so by swapping with $p_2$ and not $p_3$ we still obtain EFX allocation since in the end $p_3$ does not take items in the last iteration.
    \item $x,z,v=0$ and $l$ is odd: in this case we assign the odd $l$ item to $p_3$ and $\floor{b}{c}$ items between the ones of type $w,y$ to $p_2$. In this case holds the same thing said in the precedent point.
    \item $y,x=0$ and $k$ odd: in this case we assign the odd $k$ item to $p_3$ and $\floor{b}{c}$ items between the ones of type $z,w$ to $p_2$. In this case holds the same thing said in the precedent point.
\end{itemize}
Till now I have not considered the fact that, when $p_2$ lost an item valued $a$, since $p_2$ is not taking the items following their value for him, we have that cause of the presence of some items valued $c$ b $p_2$ in $F_2$, we could reach a point in which by adding an item $g$ we have that $v_2(F_2) \le a-b-c$ and $v_2(F_2) +v_2(g)\ge a-b$ and $v_1(F_2) \le a-b-c$ and $v_1(F_2) +v_1(g)\ge a-b$. If such thing happen, than we cannot invert the assignment and obtain an EF allocation because the bundle given to $p_1$ would have value higher than $a$ for $p_2$. Such thing can happen after that $p_2$ took value $b$ for himself and $p_1$. In order to solve such problem we can divide in the following cases:
\begin{itemize}
    \item $\exists h=(b,c,\star)\in F_2$: in this case is enough to add $g$ to $F_2$ and remove $h$ from $F_2$, after this the value for $p_2$ of $F_2$ is increased by $b-c$ while remained constant for $p_1$, so we would have that $v_2(F_2)\ge a-b-c$ and $v_1(F_2) \le a-b$, so we can unfreeze $p_1$ and move forward. 
    \item $(c,b,\star)\in F_2$: in this case, by excluding the precedent case, we have that is impossible have that $v_2(F_2) \le a-b-c$ and $v_2(F_2) +v_2(g)\ge a-b$ and $v_1(F_2) \le a-b-c$ and $v_1(F_2) +v_1(g)\ge a-b$, because $v_2(F_2) = v_2(F_2\setminus \{g\}) + b$ and $v_1(F_2) = v_1(F_2\setminus \{g\}) + c$ and $v_2(F_2\setminus \{g\}) \ge v_1(F_2\setminus \{g\})$, so we have in the end that $v_2(F_2) - v_1(F_2) \ge b-c$
    \item only $(b,b,\star), (c,c,\star)\in F_2$: in this case we can remove the items that causes the shift between multiples of $b$ and the value of $F_2$ for $p_2$ and $p_1$ (in this case $v_1(F_2) = v_2(F_2)$). These items that we remove have value lower of $b$ for $p_1$ and $p_2$ and we can assign them to $p_3$ (and $p_3$ will not take the item that he would have taken in the last iteration in which we had assigned to $p_2$ the item $g$), than we assign to $F_2$ the item $g$. So now the value of $F_2$ will be lower or equal to $a-b$. 
\end{itemize}
In all the precedent cases we have removed an item $h$ from $F_2$ without considering the value for $p_3$ and so we could think "what if that item had value higher than the ones that already took $p_3$?" This gives no problem because we already had assumed that $p_3$ could leave an item like $g$ as last item to $p_1$ or $p_2$, so if we give such item to $p_2$ and than we assign him $f$ as last item this would make no difference to assign $h$ while $p_1$ was frozen and $g$ as last item, instead if we assign $h$ as last item to $p_1$, we know that $v_3(F_3)\ge v_3(h)$, so we still have an EFX allocation. 

\paragraph{After the problematic assignment} After such problematic assignment we have a case in which we can have another problematic assignment, such case is the one in which $p_3$ has still an item valued $a$ after that $p_1$ has been unfrozen. If this happens we have that since we start from having all players that are EF towards each other except for $p_2$ that has $v_2(F_2) + c > v_2(F_1) \ge v_2(F_2)$ we have that we can obtain an EFX allocation by following the rules for such second problematic assignment. Indeed each problematic assignment has been solved by leading to an EF allocation except for the unfrozen player who can still envy the frozen one for at most $c$, so in this case is enough that if we have to chose one player to freeze between $p_1$ and $p_2$, we chose to freeze $p_2$, avoiding to let him envy for more than $c$ $p_1$. We can do such a choice since if both $p1$ and $p2$ are involved in another problematic assignment than both have that the remaining items value only $c$.